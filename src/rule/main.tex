%!TEX root = ../../rule.tex

\section{主要規則} \label{sec:main}

遊戲從起始玩家開始,並向左順時針進行,輪流進行回合並探索屋子。

為了簡單起見,所有規則中的『你』皆指正在進行動作或受到卡牌、房屋板塊規則影響的角色——探險者(包含英雄與叛徒)以及怪物。

%%%%%%%%%%%%%%%%%%%%%%%%%%%%%%%%%%%%%%%%%%%%%%%%%%%%%%%%%%%%%%%%%%%%%%%%%%%%%%%%

\subsection{屬性} \label{ssec:traits}

每位探險者皆有四個\textbf{〈屬性 Trait〉},由角色卡四邊的四排數字所表示:\textbf{〈力量 Might〉}、\textbf{〈速度 Speed〉}、\textbf{〈知識 Knowledge〉}、\textbf{〈神志 Sanity〉}。力量與速度為\textbf{〈物理屬性 Physical Traits〉},而知識與神志為\textbf{〈精神屬性 Mental Traits〉}。

許多卡牌、房間板塊、以及遊戲效果會影響你的屬性。當任何效果提升或降低你的任一屬性時,將該屬性的指標滑動相對應的格數。舉例來說,Zoe Ingstrom 的初始力量為3,若有個效果提升她3點力量時,將她的力量指標往數字大的那側滑動,指到4的位置。

每個屬性皆有其最大值,也就是角色卡上該屬性那排數字的最大值。當屬性受到效果影響而提升時,無法超過其最大值。

另外,每個屬性的最小值下方皆有一個\textbf{〈骷髏頭 Skull〉}\SkullSymbol{}標誌。在作祟發生後,當你的任一屬性降至骷髏頭標誌時,你的探險者就會死亡。在作祟發生前,沒有人會死亡——換言之,任何屬性皆不會降至骷髏頭標誌。當屬性受到效果影響而降低到或超過骷髏頭標誌時,改為停在該屬性的最小值。舉例來說,Zoe Ingstrom 的初始神志為5,若有個效果降低她2點神志時,將她的神志指標往骷髏頭的那側滑動,指到3的位置。若她再失去2點神志,除非作祟尚未發生,否則她將會死亡。

\textbf{〈傷害 Damage〉}:許多卡牌、房間板塊、以及遊戲效果會對你造成傷害。當你受到\textbf{〈物理傷害 Physical Damage〉}時,你可將該傷害自由分配到力量和速度,並對兩個屬性下降總和為該傷害的格數。同樣地,當你受到\textbf{〈精神傷害 Mental Damage〉}時,如同物理傷害的操作,但改為下降知識和神志。

附註:Zoe Ingstrom 的力量由大到小分別為『7、6、4、4、\textcolor{green!50!black}{「3」}、3、2、2、\SkullSymbol{}』,神志為「8、7、6、6、5、\textcolor{green!50!black}{「5」}、4、3、\SkullSymbol{}』,其中「$\bullet$」為起始值。

%%%%%%%%%%%%%%%%%%%%%%%%%%%%%%%%%%%%%%%%%%%%%%%%%%%%%%%%%%%%%%%%%%%%%%%%%%%%%%%%

\subsection{回合進行} \label{ssec:on-your-turn}

在你的回合當中,你可以以任意順序進行以下動作(任意次數):
\begin{itemize}
  \item \textbf{移動},
  \item \textbf{探索新房間},
  \item \textbf{使用物品}(道具牌與預兆牌),
  \item \textbf{執行擲骰},
  \item \textbf{發動攻擊}(一回合一次,僅能在作祟發生後執行)。
\end{itemize}
在作祟發生前,若你抽到預兆牌,你必須在回合結束時進行\textbf{〈\Roll{Haunt} Haunt Roll〉}。作祟發生後,遊戲將會有些許的改變(詳見\namepageref{ssec:making-a-haunt-roll})。

%%%%%%%%%%%%%%%%%%%%%%%%%%%%%%%%%%%%%%%%%%%%%%%%%%%%%%%%%%%%%%%%%%%%%%%%%%%%%%%%

\subsection{移動} \label{ssec:move}

在你的回合中,你可以移動至多等於探險者的\textbf{速度}的距離。在移動中,你進行任何行動(如使用物品或發動攻擊)。然而,當你因為任何理由而抽牌後,你本回合必須停止移動(但仍然可以繼續進行回合)。

%%%%%%%%%%%%%%%%%%%%%%%%%%%%%%%%%%%%%%%%%%%%%%%%%%%%%%%%%%%%%%%%%%%%%%%%%%%%%%%%

\subsection{探索新房間} \label{ssec:discover-a-new-room}

當你的探險者穿越一扇後面沒有房間的門時,檢查房間板塊堆最頂上的房間。如果它符合你在的樓層(\textbf{〈地面 Ground〉、〈地下 Basement〉、〈樓上 Upper〉}),將它翻開並連接在門後,然後移動至房間內,並探索房間。

放置時必須盡量合理化,門口必須對著門口,而如果無法對應所有的門的話,則造成了一些『假象』,例如被鎖死的門或是被釘死的窗戶。(這對於一間作祟的房屋而言,並不是什麼奇怪的事。)你不可以穿越鎖死的門,釘死的窗也不被算盡窗戶裡面(除非作祟劇本有特別規定)。

如果房間板塊頂的房間不符合你在的樓層,將它維持面向下的狀態放至棄牌堆。重複檢查房間板塊頂的房間直到找到符合的房間為止。

你可以穿越門並移動到相鄰的房間。所有門都是開著的,唯一的例外是\Room{Front Door},這扇門永遠都是鎖著的。你無法離開這個幢房屋或使用\Room{Front Door}(除非作祟劇本有特別規定)。所有室外的房間(如\Room{Patio})亦屬於房屋的一部分。

樓層間透過樓梯相連。\Room{Grand Staircase}永遠都連到\Room{Upper Landing};\Room{Stairs from Basement}永遠都連到\Room{Foyer}的秘門,但這扇秘門在\Room*{Stairs from Basement}被探索前皆不可使用。

有些房間有特殊符號,代表預兆牌、道具牌、事件牌(詳見\namepageref{ssec:draw-event-item-and-omen-cards})。有些房間也會有文字敘述,這些文字會在探險者進入或離開房間時發動。如果房間同時有文字與符號,先為特殊符號抽牌,再執行文字敘述的內容。

有些房間會影響移動,另外許多房間有特別的規則(詳見\namepageref{ssec:special-rooms})。

\begin{RuleBox}{我可以透過擺放新房間封鎖一個樓層嗎?}
  你無法用此方法封鎖樓層,換言之,你無法使一個樓層沒有任何未探索的門。如果所有合理的擺放皆會封鎖該樓層,棄掉該房間板塊並抽取新的直到一間不會封鎖樓層的房間為止。如果所有剩餘的房間皆會封鎖該樓層,則玩家可在最少改動下改變其他房間 的放置(例如改變方向、移往隔壁或交換位置)。

  \Room{Mystic Elevator}的移動也可能會封鎖樓層。如果\Room{Mystic Elevator}的檢定結果會導致樓層被封鎖,則電梯不會移動(就算之後可以移開電梯也不行)。
\end{RuleBox}

\begin{RuleBox}{房間板塊堆用完了怎麼辦?}
  如果你用完了房間板塊堆,將棄牌堆洗勻,並用這些牌當作新的房間板塊堆。如果你用光了一層樓的所有房間板塊,則無法繼續在該層樓探索新房間——你已經找到所有房間了。
\end{RuleBox}

%%%%%%%%%%%%%%%%%%%%%%%%%%%%%%%%%%%%%%%%%%%%%%%%%%%%%%%%%%%%%%%%%%%%%%%%%%%%%%%%

\subsection{特殊房間} \label{ssec:special-rooms}

有些房間板塊上印有規則敘述房間的效果,本章節會介紹大多數這類房間的特別規則以及介紹。這些特別的房間在房間名字上會多加一個星號(*)。

\begin{RuleBox}{\Room{Chasm}、\Room{Catacombs}、\Room{Vault}、\Room{Tower}}
  這些房間被稱為\textbf{〈屏障房間 Barrier Room〉}。屏障房間被屏障分為兩側,並且會阻擋你從一側移動到另一側。欲穿過屏障,必須進行房間板塊上寫的\textbf{\Roll{Trait}}(力量、速度、知識、神志)。每個自己的回合可執行一次這個檢定。穿過屏障不需花費行動點數。若檢定失敗,你本回合必須停止移動。你下回合仍可重新嘗試此檢定,亦可選擇回頭離開。

  探險者無法與屏障對面的其他探險者戰鬥或互動。怪物永遠忽視屏障,但若怪物結束移動於屏障房間時,其操控者必須決定牠要待在哪一側。

  若任何效果使你落入屏障房間時,你必須決定要落入哪一側。若任何效果指示你放置方形標記在此房(像是\Room{Collapsed Room}或\Event{Secret Passage}),則該標記將永久放置於你選擇的那側。
\end{RuleBox}

\begin{RuleBox}{\Room{Entrance Hall}、\Room{Foyer}、\Room{Grand Staircase}}
  \Room{Entrance Hall}、\Room{Foyer}、\Room{Grand Staircase}在同一個板塊上,但是他們被視為三間分開的房間,從其中一間房間移動至相鄰房間亦需花費1點行動點數。\Room{Grand Staircase}與\Room{Upper Landing}視為兩間分開的房間。
\end{RuleBox}

\begin{RuleBox}{\Room{Coal Chute}}
  當你進這間房間,立即滑落至\Room{Basement Landing}。滑落不需花費行動點數。角色無法結束回合於\Room*{Coal Chute}。
\end{RuleBox}

\begin{RuleBox}{\Room{Collapsed Room}}
  只有探索這間房間的探險者須執行板塊上印的速度檢定。此後,任何進入此房的探險者忽視房間的效果,並且可以選擇跳進坑洞裡。落至地下不需花費行動點數,但探險者須受到傷害。

  第一個落至地下的探險者須抽取一張地下的房間板塊(若已無未探索的地下房間,則任選一間),跌落至該處,並放置一枚\TokenSq{Below Collapsed Room}於探險者落入的房間中。

  若第一個落入地下的角色為怪物,則無需抽新的房間板塊,直接選擇一間以探索的地下房間。
\end{RuleBox}

\begin{RuleBox}{\Room{Junk Room}}
  若你在離開此房間時失去了速度屬性,且你的新速度使你沒有足夠的移動點數離開此房,你依然可以離開此房,並停留在\Room*{Junk Room}與相鄰的房間內。
\end{RuleBox}

\begin{RuleBox}{\Room{Mystic Elevator}}
  這個板塊會在你進入時立即移動。擲2顆骰子並把此板塊放置於相對應樓層並連接一扇門。如果檢定結果會導致樓層被封鎖,則電梯不會移動(就算之後可以移開電梯也不行)。若檢定結果為\Room*{Mystic Elevator}所在的樓層,依然可以移動電梯至同樓層的其他門。每個回合僅可使用\Room*{Mystic Elevator}一次。

  叛徒與怪物皆可使用\Room*{Mystic Elevator},並可無需檢定自由決定要移動至哪個樓層。然而,整個叛徒+怪物回合階段僅能使用一次電梯,並且電梯會於第一次有叛徒或怪物進入時立即移動。換言之,若叛徒在他的回合使用過\Room*{Mystic Elevator},則接下來的怪物回合電梯將不再移動。若英雄花費其整個回合於\Room*{Mystic Elevator}中並沒有移動,則他必須於回合結束時擲骰移動電梯。

  如果有多名探險者在\Room*{Mystic Elevator}內,而其中一位在移動電梯時骰到0,則所有探險者必須承受傷害。

  若有任何效果可通往\Room*{Mystic Elevator}(像是\Room{Collapsed Room}或\Event{Secret Passage}),該效果放置的標記將跟隨著電梯一同移動。
\end{RuleBox}

\begin{RuleBox}{\Room{Vault}}
  若任何效果使你落入此房(像是\Room{Collapsed Room}或\Event{Secret Passage}),你只能落入\Room*{Vault}門鎖的外側。當\Room*{Vault}被打開時,放置一枚\TokenSq{Vault Empty}於此房。叛徒仍然需要進行知識檢定來開啟\Room*{Vault}。
\end{RuleBox}

%%%%%%%%%%%%%%%%%%%%%%%%%%%%%%%%%%%%%%%%%%%%%%%%%%%%%%%%%%%%%%%%%%%%%%%%%%%%%%%%

\subsection{抽牌} \label{ssec:draw-event-item-and-omen-cards}

有些房間會有些\textbf{〈符號 Symbol〉}印在板塊上(\EventSymbol{}\ItemSymbol{}\OmenSymbol{})。當你探索一間有符號的房間時,你本回合必須停止移動(但仍然可以繼續進行回合),並抽對應符號的卡牌。只有第一個探索該房間的玩家抽牌(也只有他必須停止移動)。

如果一間房間有\textbf{〈事件 Event〉}符號(一個漩渦\EventSymbol{}),抽一張事件牌。大聲的把卡牌內容唸出,並執行上面的敘述(可能會包含擲骰檢定)。然後棄掉此牌,除非牌上另外有說明。

如果一間房間有\textbf{〈道具 Item〉}符號(一顆牛頭\ItemSymbol{}),抽一張道具牌。大聲的把卡牌內容唸出。然後將此牌放在你面前;你現在就擁有這個道具(你將攜帶或穿戴它)。你的每個回合一次,可以立即使用該道具,除非牌上另外有說明。

如果一間房間有\textbf{〈預兆 Omen〉}符號(一隻渡鴉\OmenSymbol{}),抽一張預兆牌。大聲的把卡牌內容唸出。然後將此牌放在你面前;你現在就擁有這個預兆。在你的回合結束,如果作祟尚未發生,你必須在回合結束時進行\textbf{〈\Roll{Haunt} Haunt Roll〉}(詳見\namepageref{sec:haunt})。

如果你因為房間或卡牌效果而抽新的房間板塊,且該房間上有符號,你亦需抽對應符號的卡牌。如果房間是因為其他理由被放入屋內(如作祟劇本的指示),則第一個進入該房的玩家就無需抽牌。

就算你因為抽牌而停止移動,你仍然可以進行其他行動(如使用物品)。

%%%%%%%%%%%%%%%%%%%%%%%%%%%%%%%%%%%%%%%%%%%%%%%%%%%%%%%%%%%%%%%%%%%%%%%%%%%%%%%%

\subsection{使用道具與預兆} \label{ssec:use-item-and-omen-cards}

所有的探險者皆可\textbf{使用}道具。如果作祟劇本有寫的話,某些怪物亦可使用道具。
你可以在自己的回合中的任何時候使用道具每個一次。\strong{大多數的預兆牌被視為道具:}保留這些並如同道具般使用。以下用\textbf{物品}通稱道具與視為道具的預兆。可攜帶的物品沒有上限。

在你的回合當中,任何探險者(或可攜帶物品的怪物)皆可進行以下\strong{每個}動作:
\begin{itemize}
  \item \textbf{交易物品}:與另一名在同房間的探險者交易一個物品(雙方都接受的情況下)。
  \item \textbf{棄置物品}:丟下任意數量的物品(並將一枚\TokenPenta{Item Pile}放置在該房間),其他探險者之後可以撿起任意數量的物品。
  \item \textbf{撿起物品}:從\TokenPenta{Item Pile}上撿起任意數量的物品。若\TokenPenta{Item Pile}上已無物品,棄掉該標記。
\end{itemize}
有些物品不能被交易(或搶奪),但他們可以被棄置或撿起(卡牌上會說明)。

\Omen{Bite}、\Omen{Dog}、\Omen{Girl}、\Omen{Madman}不是物品,所以他們不能被被棄置、交易、或搶奪(卡牌上亦有說明)。

探險者(或可攜帶物品的怪物)對於\strong{每個物品}每回合僅可進行以下\strong{其中一個}動作:
\begin{itemize}
  \item \textbf{使用}物品。
  \item \textbf{交易}物品。
  \item \textbf{棄置}物品。
  \item \textbf{搶奪}物品(詳見\namepageref{ssec:special-attacks})。
  \item \textbf{撿起}物品。
\end{itemize}
使用物品包含攻擊、擲骰檢定、或任何與物品相關的動作。舉例來說,一名探險者不可以在同一個回合中,使用\Omen{Spear}攻擊並交易給另一名探險者。

有些物品會影響你的屬性,若物品給予的屬性值超過該屬性的最高值,多餘的將不會曾將屬性,並自行註記超出了多少級。若之後失去該物品將少扣那麼多級屬性。舉例來說,若一個物品給予你兩級力量,但你僅需再增加一級便達到最高值,則之後失去該物品僅需扣一級力量。

\begin{RuleBox}{物品標記}
許多作祟劇本會放置一或多個\textbf{〈物品標記 Item Token〉}屋內,並且有特殊規則敘述其用途。除非劇本另外有說明,否則這些物品標記就如同道具牌與預兆牌一樣,皆可被交易、棄置、或搶奪。
\end{RuleBox}

\textbf{\Usage{Weapon}}:\Item{Axe}、\Item{Blood Dagger}、\Item{Revolver}、\Omen{Spear}、\Item{Sacrificial Dagger}屬於武器。武器僅能用於攻擊、而非防禦(詳見\namepageref{ssec:make-an-attack})。你每次攻擊僅能使用一把武器,但能夠同時攜帶多把。亦可選擇不使用武器。

\textbf{\Usage{Companion}}:\Omen{Cat}、\Omen{Dog}、\Omen{Girl}、\Omen{Madman}屬於同伴,他們會跟隨照料(擁有)他們的探險者。同伴沒有任何屬性。

\begin{RuleBox}{規則書與卡牌上的規則衝突怎麼辦?}
  請以卡牌上為準。
\end{RuleBox}

%%%%%%%%%%%%%%%%%%%%%%%%%%%%%%%%%%%%%%%%%%%%%%%%%%%%%%%%%%%%%%%%%%%%%%%%%%%%%%%%

\subsection{擲骰檢定} \label{ssec:attempt-a-die-roll}

在遊戲中的許多時刻,你會需要\textbf{〈擲骰 Roll〉}。每個骰子上皆有六個數字:0、0、1、1、2、2。

每回合的擲骰次數沒有上限。舉例來說,你可以進入一間需要擲骰的房間,同時又因為某張卡牌的效果進行另一個擲骰。然而,你一個回合不能重複進行同一個擲骰。(舉例來說,你不能一回合重複嘗試擲骰開啟\Room{Vault})

當任何卡牌、房間、或其他遊戲效果指示你擲指定數量的骰子時,依循指示並將擲出的數字加總視為\textbf{擲骰結果}。接著根據該結果執行對應的效果。

\textbf{〈\Roll{Damage} Damage Roll〉}:當任何效果寫著『受到1顆骰的物理傷害』,擲1顆骰,並降低總和為擲骰結果的力量和/或速度屬性。若效果給予多於1顆骰的傷害,同時擲該數量的骰子並將擲出的數字加總視為\Roll{Damage}結果。精神傷害亦以相同方式處理,但改為下降知識和神志。

\textbf{〈\Roll{Trait} Trait Roll〉}:有些卡牌、房間、及作祟劇本會指示你根據你的探險者屬性(力量、速度、知識、神志)擲骰。此時,擲等同於該屬性數值數量的骰子。舉例來說,若你的探險者須進行神志檢定,且她的神志為4,則擲4顆骰子並將擲出的數字加總視為神志檢定結果。無論成功或失敗,該效果會指示你根據結果執行對應的效果。攻擊及防禦(\Roll{Combat})並非\Roll{Trait},就算他們牽涉到力量或其他屬性亦否(詳見\namepageref{ssec:make-an-attack})。

\textbf{〈\Roll{Task} Task Roll〉}:某些作祟劇本會指示你完成某些特定的任務(像是驅邪),你每回合僅能執行該檢定一次,就算該檢定能以多個屬性完成亦是如此(像是可用知識或神志檢定的驅邪一回合也只能檢定一次)。

附註:\textbf{〈檢定 Roll〉}泛指遊戲中\textbf{〈擲骰 Roll〉},可視為同義詞(英文即為同一個字)。

%%%%%%%%%%%%%%%%%%%%%%%%%%%%%%%%%%%%%%%%%%%%%%%%%%%%%%%%%%%%%%%%%%%%%%%%%%%%%%%%

\subsection{發動攻擊} \label{ssec:make-an-attack}

\textbf{在作祟發生前無法發動攻擊!}

每回合一次,你可以對一名在同房間的\textbf{〈對手 Opponent〉}發動攻擊。(對手指的是那些想阻止你移動或是妨礙你的探險者或怪物。)當你發動攻擊時,你與對手分別擲等同於自己力量數量的骰子。擲骰結果較大者擊敗其對手,並造成等同於兩者擲骰結果差值的\textbf{物理傷害}。(舉例來說,若你擲出6而你的對手擲出5,則你對其造成1點物理傷害。)若平手,則雙方皆不會受到傷害。

有些效果讓你可以用力量以外的屬性進行攻擊。攻擊方式與力量攻擊相同,除了雙方皆根據該屬性擲骰而非力量。舉例來說,若你發動速度攻擊,則你與對手皆以自己的速度擲骰。速度攻擊亦造成物理傷害。神志與知識攻擊則造成\textbf{精神傷害}。

你不能用某種屬性攻擊沒有該屬性的對手。舉例來說,如果一個怪物沒有神志屬性,則你無法對牠發動神志攻擊。

當你擊敗對手時,你亦可選擇不傷害對手而做些別的事,例如搶奪物品(詳見\namepageref{ssec:special-attacks})。

怪物被擊敗時僅會被\textbf{〈震懾 Stun〉}而不會死亡,除非作祟劇本有特別規定(詳見\namepageref{ssec:how-monsters-work})。你依然可以攻擊被震懾的怪物,如果這樣做對你有利的話(例如搶奪物品或是用特殊物品消滅牠)。被震懾的怪物仍然需擲骰防禦,但這種情況下英雄若被擊敗也不會受傷。

你每回合可以同時進行\textbf{作祟相關行動}(在作祟劇本中提到的動作)以及發動攻擊。

%%%%%%%%%%%%%%%%%%%%%%%%%%%%%%%%%%%%%%%%%%%%%%%%%%%%%%%%%%%%%%%%%%%%%%%%%%%%%%%%

\subsection{特殊攻擊} \label{ssec:special-attacks}

\textbf{〈遠程攻擊 Distance Attack〉}:\Item{Revolver}是遠程攻擊的代表。它允許你攻擊一位在\textbf{〈視線 Line of Sight〉}內房間(從你所在的房間往任何方向,透過任何未受阻礙的門直線延伸的所有房間)的對手。你若被擊敗也不會受傷。某些怪物亦可發動遠程攻擊。

\textbf{〈搶奪物品 Stealing Item〉}:當你擊敗任何對手並造成2點物理傷害,你可選擇不造成傷害而\textbf{〈搶奪 Steal〉}一個可被交易的物品。發動遠程攻擊時無法搶奪物品。有些作祟劇本亦有關於搶奪的特殊規則。

\begin{RuleBox}{戰鬥範例}
  假設你的探險者,Jenny LeClerc,正在攻擊一隻狼人。她的力量為4,故她擲4顆骰攻擊。假設你擲出了5,而叛徒替狼人擲出了8!Jenny因此受到了3點物理傷害。假設你選擇將力量降低2級(至3),並將速度降低1級(依然為4)。Jenny依然活著,但她受傷了!

  附註:Jenny LeClerc 的力量由大到小分別為『8、6、5、4、4、\textcolor{green!50!black}{「4」}、4、3、\SkullSymbol{}』,速度為『8、6、5、4、\textcolor{green!50!black}{「4」}、4、3、2、\SkullSymbol{}』,其中「$\bullet$」為起始值。
\end{RuleBox}
