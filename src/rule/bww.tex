%!TEX root = ../../rule.tex

\section{寡婦望} \label{sec:bww}

%%%%%%%%%%%%%%%%%%%%%%%%%%%%%%%%%%%%%%%%%%%%%%%%%%%%%%%%%%%%%%%%%%%%%%%%%%%%%%%%

\subsection{新卡牌及板塊}

在遊戲開始前,將擴充的事件卡、道具卡、預兆卡、房間板塊與基礎遊戲的一起洗勻。

%%%%%%%%%%%%%%%%%%%%%%%%%%%%%%%%%%%%%%%%%%%%%%%%%%%%%%%%%%%%%%%%%%%%%%%%%%%%%%%%

\subsection{術語表更新}

\paragraph{〈同伴 Companion〉}
\Omen{Cat}、\Omen{Dog}、\Omen{Girl}、\Omen{Madman}——這幾張預兆卡屬於同伴,他們會跟隨照料(擁有)他們的探險者。同伴沒有任何屬性。

\paragraph{〈室外 Outside〉}
\Room{Balcony}、\Room{Conservatory}、\Room{Gardens}、\Room{Graveyard}、\Room{Patio}、\Room{Roof Landing}、\Room{Solarium}、\Room{Tower}、\Room{Tree House}、\Room{Widow's Walk}屬於室外。

\paragraph{〈武器 Weapon〉}
道具\Item{Axe}、\Item{Blood Dagger}、\Item{Boomstick}、\Item{Chainsaw}、\Item{Revolver}、\Item{Sacrificial Dagger}及預兆\Omen{Spear}屬於武器。

\paragraph{〈窗戶 Windows〉}
\Room{Bedroom}、\Room{Chapel}、\Room{Dining Room}、\Room{Drawing Room}、\Room{Grand Staircase}、\Room{Master Bedroom}、\Room{Sewing Room}有窗戶。

%%%%%%%%%%%%%%%%%%%%%%%%%%%%%%%%%%%%%%%%%%%%%%%%%%%%%%%%%%%%%%%%%%%%%%%%%%%%%%%%

\subsection{頂樓}

〈頂樓 Roof〉是一個新的樓層。在遊戲開始前,將\Room{Roof Landing}放入屋內。任何背面寫有樓上或頂樓的房間接可以放在頂樓。頂樓並非第二個樓上;舉例來說,當你使用\Room{Mystic Elevator}擲到4時你只能前往頂樓。

%%%%%%%%%%%%%%%%%%%%%%%%%%%%%%%%%%%%%%%%%%%%%%%%%%%%%%%%%%%%%%%%%%%%%%%%%%%%%%%%

\subsection{平台}

\Room{Basement Landing}、\Room{Grand Staircase}/\Room{Foyer}/\Room{Entrance Hall}、\Room{Upper Landing}、\Room{Roof Landing}皆為\Room{Landing}。

%%%%%%%%%%%%%%%%%%%%%%%%%%%%%%%%%%%%%%%%%%%%%%%%%%%%%%%%%%%%%%%%%%%%%%%%%%%%%%%%

\subsection{新符號}

\DumbwaiterSymbol{}符號代表該房間有個升降機。你可以使用升降機並多耗1點行動點數移動到往上一層或往下一層的\Room{Landing}。使用升降機除了需多耗1點行動點數,亦須遵守其他移動規則。舉例來說,若\Room{Menagerie}在地下,且你欲從該房透過升降機移動到\Room{Foyer},則你需要消耗2點行動點數;若\Room{Menagerie}在地面,則你可以消耗2點行動點數移動到\Room{Basement Landing}或\Room{Upper Landing}。若你沒有足夠的行動點數,則無法使用升降機。

\QuestionSymbol{}符號代表你探索該房時可以抽取任何一種卡牌。

%%%%%%%%%%%%%%%%%%%%%%%%%%%%%%%%%%%%%%%%%%%%%%%%%%%%%%%%%%%%%%%%%%%%%%%%%%%%%%%%

\subsection{探險者標記}

共有36個、6色的探險者標記,用來標示該探險者擁有某物或完成某事。當你使用\Room{Chapel}、\Room{Gymnasium}、\Room{Larder}、\Room{Library}、\Room{Menagerie}、\Room{Study}房間的能力後,將你的探險者標記放置在那個房間。之後你再進入該房間時,便無法再使用房間的能力。探險者標記亦可用於作祟劇本。

%%%%%%%%%%%%%%%%%%%%%%%%%%%%%%%%%%%%%%%%%%%%%%%%%%%%%%%%%%%%%%%%%%%%%%%%%%%%%%%%

\subsection{障礙物及鎖}

\TokenSq{Obstacle}與\TokenSq{Lock}會阻礙探險者移動,欲通過則必須達成某些特殊條件。除非作祟劇本額外敘述,否則叛徒與怪物可忽視這些阻礙標記自由移動。

%%%%%%%%%%%%%%%%%%%%%%%%%%%%%%%%%%%%%%%%%%%%%%%%%%%%%%%%%%%%%%%%%%%%%%%%%%%%%%%%

\subsection*{}

\begin{RuleBox}{選用規則:只玩新的作祟劇本}
  若你在一個基礎房間因一個基礎預兆觸發作祟,則將會是一個基礎作祟劇本。如果你只想玩擴充的作祟劇本,你可以棄掉該張預兆卡改抽下一張,直到找到一個擴充的預兆卡。或是你也可以棄掉該房間板塊改抽下一個,直到找到一個擴充的房間板塊。若所有的預兆卡及房間板塊都已經在遊戲內了,則選擇最晚放入遊戲的擴充預兆卡或房間板塊。
\end{RuleBox}
