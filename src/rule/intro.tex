%!TEX root = ../../rule.tex

\section{山中小屋} \label{sec:intro}

在山中小屋中,每個玩家選擇一位\textbf{〈探險者 Explorer〉}來探索這個令人毛骨悚然的老舊房屋。當你探索這個房屋,你會發現新的房間。每次你進到一間新的房間,你將可能發現些什麼…或是有什麼會發現你。影響遊戲的進行,取決於探險者對於房屋中的各個驚喜的反應(無論好壞)。每次遊玩這個遊戲,房屋皆會有所不同。

在遊戲中的某個隨機時刻,其中一位探險者將會觸發一個劇本——稱為\textbf{〈作祟 Haunt〉}。當作祟被揭露時,其中一名探險者將成為\textbf{〈叛徒 Traitor〉},全力對抗他原來的夥伴。而其餘的探險者將成為\textbf{〈英雄 Hero〉},為了自己的生存而奮鬥。從此刻起,遊戲變成了叛徒與英雄間的戰鬥——通常至死方休。

這款遊戲有 50 個作祟劇本,每個皆有不同的故事。

%%%%%%%%%%%%%%%%%%%%%%%%%%%%%%%%%%%%%%%%%%%%%%%%%%%%%%%%%%%%%%%%%%%%%%%%%%%%%%%%

\subsection{遊戲目標}

探索這棟房屋,並努力使你的探險者更加強大,直到作祟發生。在那之後,無論你是叛徒還是英雄,你的目標皆是完成你那一側的勝利條件。

%%%%%%%%%%%%%%%%%%%%%%%%%%%%%%%%%%%%%%%%%%%%%%%%%%%%%%%%%%%%%%%%%%%%%%%%%%%%%%%%

\subsection{初始設置}

\begin{itemize}
  \item \textbf{把〈叛徒手冊 Traitor’s Tome〉和〈生存指南 Secrets of Survival〉擺到一邊。}在作祟發生後,你才會需要他們。
  \item \textbf{每位玩家選擇一張角色卡。}卡片兩面各有一名角色,選擇其中一面。
  \item \textbf{在角色卡片四邊放上塑膠指標夾},並對其角色各屬性的綠色起始數值。
  \item \textbf{洗牌。}將所有事件牌面朝下洗勻,並放置在一個大家都拿得到的地方。對道具牌、預兆牌也做同樣的事情。
  \item 找到\textbf{\Room{Basement Landing}、\Room{Entrance Hall}/\Room{Foyer}/\Room{Grand Staircase}、\Room{Upper Landing},}並將他們由左至右擺開,中間相隔一定的距離。
  \item \textbf{將其餘的房間板塊面朝下洗勻並疊成一落,}無需在意背面的圖示。
  \item \textbf{每位玩家將其探險者人物放置到\Room{Entrance Hall},}探險者人物與角色卡的顏色相同。
  \item \textbf{將骰子堆在一起,}並放置在一個大家都拿得到的地方。
  \item \textbf{決定起始玩家。}角色卡上有寫該角色的生日,生日最近的探險者為起始玩家,回合向左順時針進行。
\end{itemize}
