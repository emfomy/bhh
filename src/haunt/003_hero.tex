%!TEX root = ../../haunt_hero.tex

\chapter{蛙腿燉肉 Frog-Leg Stew}

\begin{HauntStory}
  刺耳的咯笑聲在屋內迴響。
  \begin{quote}
    『不、不、不!別躲避我,我頑皮的小猴子們!你們這些壞壞的小蟾蜍,竟敢偷走奧瑪的書。小壞蛋!奧瑪恐怕得把你們的小鼻子打斷…或者更糟…比那個還糟…』
  \end{quote}
\end{HauntStory}

\HauntSection{對於壞人你知道}
\Monster{Witch}施展了某種無敵法術,並且她還可以將人們變成\Monster{Frog}。

\HauntSection{你獲勝於…}
…你殺死\Monster*{Witch}時。

\HauntSection{如何殺死\Monster*{Witch}}
你必須使用法術書(即\Omen{Book}預兆牌)對\Monster*{Witch}施展〈凡人之形 Form of Mortal〉法術,使女巫不再無敵並且可以被攻擊;此法術需要\Monster{Mandrake Root}(用\TokenPenta{Item}表示)。

欲殺死\Monster*{Witch},你必須依照順序完成以下三個步驟。每位英雄在其回合皆可進行其中一項步驟。
\begin{itemize}
  \item 找出\Monster*{Mandrake Root}。如果你探索到了一間含有\Monster*{Mandrake Root}的房間,叛徒會將一枚標記放進該房間中。\Monster*{Mandrake Root}亦可能存在於已探索的房間中。
  \item 如果你身處於有\Monster*{Mandrake Root}的房間中,你可以進行知識檢定達 4+ 嘗試將其挖掘出來。如果你成功了,將該標記放置在你的角色卡上。
  \item 如果你同時持有\Monster*{Mandrake Root}及\Omen{Book},並且與\Monster*{Witch}處於同一房間中,你可以進行知識檢定達 6+ 嘗試施展凡人之形法術。如果你成功了,自此你便可正常地攻擊\Monster*{Witch};並且任何成功的攻擊皆可殺死\Monster*{Witch}。如果你失敗了,你依然保有\Monster*{Mandrake Root},並且可於下個回合繼續嘗試。
\end{itemize}

\vfill\null\pagebreak

\HauntMobStatusSection{\Monster{Frog}}
\vspace*{-1em}
\begin{itemize}
  \item 當英雄變成\Monster*{Frog}時,丟下持有的所有物品及同伴卡牌,並將其力量及知識屬性降至最低數值。\Monster*{Frog}不可攻擊、抽牌、或探索新房間。其他不是\Monster*{Frog}的探險者可如同物品般攜帶\Monster*{Frog}(\Monster*{Frog}在被攜帶時無法進行任何行動)。
  \item 如果你持有\Omen{Book},並且與\Monster*{Frog}處於同一房間中,你可以進行知識檢定達 4+ 嘗試施展法術將\Monster*{Frog}變回人類,此法術亦將該英雄之所有屬性恢復至初始值。
\end{itemize}

\HauntSection{特殊攻擊規則}
\vspace*{-1em}
\begin{itemize}
  \item 在你施展凡人之形法術之前,\Monster*{Witch}都會是無敵的,並且無法被攻擊。
  \item 當\Monster{Cat}出現後,探險者可以攻擊牠。
\end{itemize}

\HauntSection{如果你獲勝…}
\begin{HauntStory}
  女巫尖叫著:
  \begin{quote}
    『不不不不不不!你不能這麼做!叫他們住手,我的小糖糖!你們會後悔的!我會爬進你們的夢魘中並放光你們的鮮血!你們的腦袋將會癢到令你在上面搔出一個洞讓東西流出來!我會…』
  \end{quote}
  就在你準備抓起一盞檯燈砸向她的頭以停止她刺耳的聲音時,她消去了…至少現在是。
\end{HauntStory}
