%!TEX root = ../../haunt_hero.tex

\chapter{The Web of Destiny 命運蛛網}

\begin{HauntStory}
  這蜘蛛網大到令你的心靈拒絕去看它。現在你被他纏住了,你的臉和身體被壓進它黏黏的絲裡,並為了對抗你的皮膚而漸漸變硬。如果你不快點離開,便將永遠無法逃脫。在你視野的邊緣,你看見一塊影子從天花板脫離…喔不,不是影子…是一隻蜘蛛,滑進了蜘蛛網。牠爬到你身上,然後你的胃感覺像是著了火。向下一看,你看見一根螫針刺入你的腹部。你放身尖叫…但是會有人聽到嗎?
\end{HauntStory}


\HauntSection{此刻}
\vspace*{-1em}
\begin{itemize}
  \item 作祟揭露者被糾纏在一個黏黏的蜘蛛網中。該角色成為了\textbf{〈受困的探險者〉}。受困的探險者無法移動,但是依然可以攻擊蜘蛛網以摧毀它。他亦可使用及交易物品。
  \item 如果\Item{Medical Kit}尚未被探索,則下次當有英雄欲抽取道具牌時,皆可改為直接搜尋道具牌堆並抽將其取出,然後洗勻牌堆。
  \item 將等同於玩家數量的\TokenTri{Might Roll}放置在一旁備用。
  \item 受困的探險者已被巨大的蜘蛛卵寄生。最終,它們將會孵化…
\end{itemize}

\HauntSection{對於壞人你知道}
一隻異常巨大的\Monster{Spider}已經甦醒,他想要保護受困的探險者,直到牠的卵孵化。

\HauntSection{你獲勝於…}
…受困的探險者被救出,蜘蛛卵被摧毀,且至少一位探險者逃出房屋時。

\vfill\null\pagebreak

\HauntSection{如何摧毀蜘蛛網及卵}
在蜘蛛卵尚未被摧毀前,被受困的探險者的任何屬性皆不會降至骷髏頭標誌。
\begin{itemize}
  \item 你可以使用力量攻擊攻擊蜘蛛網,它會以力量4進行防禦。如果你擊敗它,在房間內放置一枚\TokenTri{Might Roll}(而不是造成傷害)。你就算被蜘蛛網擊敗也不會受傷。當該房間內有等同於玩家數量的\TokenTri{Might Roll}時,蜘蛛網即被摧毀。
  \item 如果你持有\Item{Medical Kit},並且和受困的探險者身處同一個房間內,你可以進行知識檢定達 4+ 嘗試摧毀蜘蛛卵。如果你擁有\Item{Healing Salve},你可以使用它直接摧毀蜘蛛卵而無需進行知識檢定。
\end{itemize}

\HauntSection{如何逃出房屋}
在受困探險者被救出,且蜘蛛卵被摧毀之後,英雄們即可開始逃出房屋。你可以進行知識檢定達 6+(開鎖)、或是進行力量檢定達 6+(破壞鎖頭)嘗試開啟\Room{Entrance Hall}的大門,如果你成功,抽取一張事件牌並且結束你的回合。在接下來的回合,任何存活的英雄皆可花費1點行動點數從\Room{Entrance Hall}逃出房屋。

\HauntSection{如果你獲勝…}
\begin{HauntStory}
  撥開眼前的蜘蛛網,你跌跌撞撞地跑出莊園。回頭一看,你看見上方的窗戶閃爍著光線。你只能一邊顫抖、一邊跨出步伐。一步…然後再一步。

  \begin{quote}
    是時候離開了…立刻…
  \end{quote}
\end{HauntStory}
