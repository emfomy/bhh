%!TEX root = ../../haunt_hero.tex

\chapter{I Was a Teenage Lycanthrope 我是少年狼}

\begin{HauntStory}
  一聲尖叫劃破寂靜,聲音越來越大且悽慘,嚇的你都開始想尖叫了。正當你覺得你再也無法忍受時,尖叫聲開始變得顫抖且低沈,直到化為純粹憤怒的嚎叫。你的影子開始發抖,因為你了解到你正沐浴在滿月的月光之下⋯
\end{HauntStory}

\HauntSection{此刻}
將一枚\TokenPenta{Item}(代表\Monster{Silver Bullets})放置在一旁備用。

\HauntSection{對於壞人你知道}

叛徒是一隻\Monster{Werewolf},而且會越變越強大。
\Monster*{Werewolf}可以可以傳染獸化病給其它人,並且把他們也變成\Monster*{Werewolf}。

\HauntSection{你獲勝於…}
…所有\Monster*{Werewolf}皆死亡時。你不一定要殺死\Omen{Dog}。

\HauntSection{如何殺死\Monster*{Werewolf}}
你必須要找到\Item{Revolver}並且製造出\Monster{Silver Bullets}。為此,你必需完成下列步驟。每一位英雄在一回合中只能嘗試其中一個步驟。
\begin{itemize}
  \item 如果你沒有\Item{Revolver},你可以在\Room{Attic}、\Room{Game Room}、\Room{Junk Room}、\Room{Master Bedroom}、\Room{Vault}進行知識檢定達 5+ 嘗試找出它。如果檢定成功,搜尋道具牌堆並抽將其取出,然後洗勻牌堆。你可以在同一間房間內進行多次搜索,但是你每回合中只能搜索一次。
  \item 到\Room{Research Laboratory}或\Room{Furnace Room}進行知識檢定達 5+ 嘗試製造\Monster{Silver Bullets}。在一位探險者努力製造\Monster*{Silver Bullets}的同時另一位探險者可以去尋找\Item*{Revolver}。(這兩件事可以依任何順序完成。)
  \item 將\Item*{Revolver}與\Monster*{Silver Bullets}交給同一位英雄。
  \item 你可使用\Item*{Revolver}射擊\Monster*{Silver Bullets}殺死\Monster*{Werewolf}或\Omen*{Dog}(詳見特殊攻擊規則)。
\end{itemize}

\vfill\null\pagebreak

\HauntSection{你在你的回合必須…}
如果你被\Monster*{Werewolf}或\Omen*{Dog}攻擊且受傷的話,你會被咬而且有可能被獸化病詛咒影響。在接下來的回合開始時,你必須進行神志檢定 4+ 以抵抗詛咒,如果檢定失敗,你變成\Monster*{Werewolf}且不再是英雄。(你必須閱讀\strong{叛徒手冊}並立即執行\strong{此刻}章節的內容。)

在\Monster*{Werewolf}都被殺死時,被咬傷的英雄若成功抵禦詛咒而沒有變成\Monster*{Werewolf}的話,他依然獲勝⋯至少到下次月圓為止。

\HauntSection{特殊攻擊規則}
當英雄使用\Monster*{Silver Bullets}擊敗\Monster*{Werewolf}或\Omen*{Dog},\Monster*{Werewolf}或\Omen*{Dog}立即死亡。(\Item*{Revolver}的子彈永遠不會用完。)

\HauntSection{如果你獲勝…}
\begin{HauntStory}
  雲朵掠過滿月,遮擋其光芒。站在你已故朋友的破爛屍體旁邊,房屋顯得昏暗且寧靜。你知道你必須這麼做才能存活⋯但在了解自己所做所為之後,未來該如何自處呢?
\end{HauntStory}
