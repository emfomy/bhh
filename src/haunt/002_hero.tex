%!TEX root = ../../haunt_hero.tex

\chapter{Séance 降魂儀式}

\begin{HauntStory}
  屋內充滿了令人毛骨悚然的寒氣,迷霧緩慢地從地面盤旋而上。一個聲音於空中響起:
  \begin{quote}
    『我必須安息…讓我的靈魂安息…否則死路一條…』
  \end{quote}
\end{HauntStory}

\HauntSection{此刻}
將等同於玩家數量的\TokenTri{Knowledge Roll}、等同於玩家數量的\TokenTri{Sanity Roll}、一枚\TokenPenta{Item}(代表屍骸)、一枚\TokenMon{purple}(代表\Monster{Ghost})放置在一旁備用。

\HauntSection{對於壞人你知道}
叛徒正趕著在你們之前召喚\Monster*{Ghost}。

\HauntSection{你獲勝於…}
…在叛徒掌控\Monster*{Ghost}之後擊敗祂,或在你召喚出\Monster*{Ghost}之後埋葬祂。

\HauntSection{如何召喚\Monster*{Ghost}}
你們與叛徒必須比賽誰先召喚\Monster*{Ghost}。你必須先執行降魂儀式以召喚\Monster*{Ghost}:
\begin{itemize}
  \item 英雄可以在\Room{Pentagram Chamber}內進行知識或神志檢定達 5+ 嘗試執行降魂儀式。每個回合僅能進行一種檢定。
  \item 每當檢定成功,(依據該檢定屬性)將一枚\TokenTri{Knowledge Roll}或\TokenTri{Sanity Roll}放進房間內。當英雄方放置了等同於玩家人數一半(向下捨去)的標記,他們便完成降魂儀式。
  \item 如果英雄方在叛徒之前完成降魂儀式,他們即掌控\Monster*{Ghost}(依照下一節的指示)。如果叛徒成功完成降魂儀式,則會是由他掌控\Monster*{Ghost}。
\end{itemize}

\vfill\null\pagebreak

\HauntSection{如何你先召喚\Monster*{Ghost}…}
\Monster*{Ghost}宣布(請大聲說出):
\begin{quote}
  『埋葬我的屍骨!』
\end{quote}
\vspace*{-1em}
\begin{itemize}
  \item 將\Monster*{Ghost}標記放置於降魂儀式成功的房間內,他會待在那邊直到你失去祂的掌控權。
  \item 將回合/傷害記錄器設置好並指到1。隨後,每當完成降魂儀式的玩家結束回合時,將記錄器的數值推進到下一個數字。你必須在第5回合(數字指到5)開始前埋葬屍骨。
  \item 每回合一次,你可以在\Room{Attic}、\Room{Bedroom}、\Room{Master Bedroom}內執行知識檢定達 5+ 嘗試尋找屍骨。如果檢定成功,將屍骸標記放置於你的角色卡上。
  \item 將屍骸帶到\Room{Crypt}或\Room{Graveyard},並執行知識檢定達 5+ 已找到正確的墓碑以埋葬屍骨。
  \item 當你完成此事,\Monster*{Ghost}將不再可以攻擊。如果你在第5回合開始時仍未成功埋葬屍骨,則叛徒將掌控\Monster*{Ghost},並依照叛徒手冊的指示控制\Monster*{Ghost}。若此事發生,則埋葬屍骨已不足以安撫\Monster*{Ghost},你必須消滅祂。
\end{itemize}

\HauntSection{特殊攻擊規則}
\vspace*{-1em}
\begin{itemize}
  \item 在降魂儀式完成前任何人皆不可出手攻擊。
  \item 當叛徒成功掌控\Monster*{Ghost},你可以(且只能)對祂進行神志攻擊,但僅限於使用\Omen{Ring}、或是當你與\Monster*{Ghost}同處於\Room{Pentagram Chamber}內時。一次成功的神志攻擊即可消滅\Monster*{Ghost}。
  \item 若\Monster*{Ghost}對英雄進行攻擊但卻反遭擊敗,祂不會受傷。
\end{itemize}

\HauntSection{如果你獲勝…}
\begin{HauntStory}
  迷霧緩緩散去,籠罩在你心頭的駭人寒意亦消退。一種平靜和滿足的感覺油然而生。一個靈魂已獲得安息。
\end{HauntStory}
