%!TEX root = ../../haunt_traitor.tex

\chapter{Séance 降魂儀式}

\begin{HauntStory}
  屋內充滿了令人毛骨悚然的寒氣,迷霧緩慢地從地面盤旋而上。一個聲音於空中響起:

  \begin{quote}
    『我必須安息…讓我的靈魂安息…否則死路一條…』
  \end{quote}

  當語句消散,你手上的靈魂版開始隨著你的心跳的節拍悸動。低頭望像板子,你看見霧氣在其表面盤旋成字:

  \begin{quote}
    『殺 光 他 們』
  \end{quote}
\end{HauntStory}

\HauntSection{此刻}
\vspace*{-1em}
\begin{itemize}
  \item 你的探險者仍在遊戲內,但是變成了叛徒。
  \item 將一枚\TokenTri{Knowledge Roll}、一枚\TokenTri{Sanity Roll}、一枚\TokenMon{purple}(代表\Monster{Ghost})放置在一旁備用。
  \item 若\Room{Pentagram Chamber}不在屋內,將其找擺放到地下室,並且距離你五格以上的房間。如果沒有距離五格以上的房間,則擺放在該樓層盡量越遠越好。然後洗勻排堆。
\end{itemize}

\HauntSection{對於英雄你知道}
他們正趕著在你之前召喚\Monster*{Ghost}。如果他們成功了,他們將掌控\Monster*{Ghost}並且獲得一項任務去完成。如果他們失敗,他們將改為消滅\Monster*{Ghost}。

\HauntSection{你獲勝於…}
…所有英雄皆死亡時,無論是誰先召喚出\Monster*{Ghost}。

\HauntSection{如何召喚\Monster*{Ghost}}
你與英雄們必須比賽誰先召喚\Monster*{Ghost}。你必須先執行降魂儀式以召喚\Monster*{Ghost}:
\begin{itemize}
  \item 當你持有\Omen{Spirit Board},你可以進行進行知識或神志檢定達 5+ 嘗試執行降魂儀式。每個回合僅能進行一種檢定。每當檢定成功,(依據該檢定屬性)拿取一枚\TokenTri{Knowledge Roll}或\TokenTri{Sanity Roll}。當你成功拿取兩種標記(各一枚)時,你便完成降魂儀式。
  \item 如果你在英雄之前完成降魂儀式,你即掌控\Monster*{Ghost}(依照下一節的指示),並將\Monster*{Ghost}標記放置你所在的房間內。如果英雄們成功完成降魂儀式,他們會告知你發生了什麼事。
\end{itemize}

\HauntSection{如何你先召喚\Monster*{Ghost}…}
\Monster*{Ghost}宣布(請大聲說出):
\begin{quote}
  『我將對生者復仇!』
\end{quote}

\vfill\null\pagebreak

\HauntMobStatusSection{\Monster{Ghost}}
\HauntMobStatus{速度4\quad{}神志6}
\begin{itemize}
  \item 若你搶在英雄之前完成降魂儀式,或是他們先完成了降魂儀式卻在任務中失敗了,你皆會掌控\Monster*{Ghost}。若你的探險者死亡,你依然繼續掌控\Monster*{Ghost}。
  \item \Monster*{Ghost}每回合必須向一名英雄移動,並且盡可能地進行攻擊。
  \item 在\Monster*{Ghost}的第一個回合結束時起,這棟房屋開始崩塌。第一個崩塌的房間必定是\Room{Attic},若\Room*{Attic}不存在,則選擇任何一個沒有人的樓上房間。隨後,在每個探險者的回合結束時,要求他選擇一間房間進行崩塌。
  \item 每當房間崩塌時,將其翻面並以黑色面朝上。只有當該房間鄰接著一間已崩塌的房間時才可以崩塌,鄰接的房間不用有門相連亦可。房間崩塌時,該房間內的所有探險者立即死亡(包括你)。
  \item 當所有樓上的房間皆崩塌時,從\Room{Grand Staircase}開始崩塌地面的房間(並使用標記來代表\Room*{Grand Staircase}、\Room*{Foyer}、\Room*{Entrance Hall}的崩塌)。當所有地面的房間皆崩塌時,從\Room{Basement Landing}開始崩塌地下的房間。
  \item 若\Room{Mystic Elevator}的目標樓層中沒有含有未探索門扉的未崩塌房間的話,則它就不會移動。
  \item 只有\Monster*{Ghost}可以通過崩塌房間。\Monster*{Ghost}亦可穿過牆壁移動,但不可穿過天花板及地板。
\end{itemize}

\HauntSection{特殊攻擊規則}
\vspace*{-1em}
\begin{itemize}
  \item 在降魂儀式完成前任何人皆不可出手攻擊。\iffalse{如果英雄控制\Monster*{Ghost},他們會告知你特殊攻擊規則。}\fi
  \item \Monster*{Ghost}以神志進行攻擊,並造成精神傷害。只有持有\Omen{Ring}、或是與\Monster*{Ghost}同處於\Room{Pentagram Chamber}內的英雄可以(且只能)使用神志攻擊\Monster*{Ghost}。
  \item 若\Monster*{Ghost}對英雄進行攻擊但卻反遭擊敗,祂不會受傷。
\end{itemize}

\HauntSection{如果你獲勝…}
\begin{HauntStory}
  現在迷霧從上到下充滿了屋子。你滑過它們,就像在你身邊盤旋的鬼魂一樣安靜。你的心跳減慢且平靜。安靜。現在,有兩個靈魂一起在此地作祟。直到永遠。
\end{HauntStory}
