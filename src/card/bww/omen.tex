%!TEX root = ../../../card.tex
%
\begin{OmenCard}{bww}{Bloodstone}{}
  \begin{CardStory}
    深綠森林沉寂的午夜裡,\\
    有著深紅色脈動的石頭。\\
    人們說,你無法從石頭汲取血液,\\
    看來他們指的不是這一顆⋯
  \end{CardStory}
  每回合一次,當你進行任何\Roll{Trait}前,可摩擦\ThisName{}並犧牲1級任意屬性,多擲2顆骰(至多8顆)。\smallbreak
\end{OmenCard}%
\linebreak[0]%
\begin{OmenCard}{bww}{Box}{}
  \begin{CardStory}
    人們老是告訴你要跳脫框架思考,\\
    但你現在該擔心的是會死在裡面⋯
  \end{CardStory}
  當抽到\ThisName{}時,再抽取1張道具卡放在此卡上面。\smallbreak
  \ThisName{}內的物品不可被棄置或搶奪。當有人要搶奪這個道具,你可以棄掉\ThisName{}以避免該物品被奪走。\smallbreak
  在你的回合中,你可將\ThisName{}內的物品棄掉,並將另一個你持有的物品放進\ThisName{}。若\ThisName{}是空的,你可以直接將一個你持有的物品放進\ThisName{}。\smallbreak
  \ThisName{}不可被搶奪。\smallbreak
\end{OmenCard}%
\linebreak[0]%
\begin{OmenCard}{bww}{Cat}{companion}
  \begin{CardStory}
    一般認為被貓穿越正要行徑的路線\\
    是不幸的預兆,不過換個角度想,\\
    也許是你穿越牠正要行徑的路線,\\
    而牠似乎因此有點不滿。\\
    不過現在牠正陪伴著你。
  \end{CardStory}
  當你進行任何\Roll{Trait}前,你可以把\ThisName{}叫出來碰碰運氣。擲1顆骰子:
  \begin{itemize}
    \item[1+] 擲出的結果增加2點。
    \item[0] 擲出的結果減少2點。
  \end{itemize}
  \ThisName{}不可被棄置、交易、或搶奪。\smallbreak
\end{OmenCard}%
\linebreak[0]%
\begin{OmenCard}{bww}{Key}{}
  \begin{CardStory}
    你曾經好奇的想著,要怎麼\\
    將這屋子裡所有的門鎖都打開?\\
    但問題是,它們為什麼\\
    在一開始就被鎖上了呢?
  \end{CardStory}
  \footnotesize
  當你要從卡牌或是房間效果上檢定以打開或拿取東西時,可以使用\ThisName{}並多擲4顆骰(至多8顆)。(像是\Room{Locked Room}、\Event{Locked Safe}、\Room{Vault})\smallbreak
  你可以直接進出有\TokenSq{Lock}的房間。\smallbreak
  在你的回合開始時,若你在一扇未上鎖的門旁邊,你可以將1枚\TokenSq{Lock}放在該門上。\smallbreak
  欲通過有\TokenSq{Lock}的門,必須進行知識檢定達 3+ 嘗試解鎖,若成功,移除標記並正常通過該門。
\end{OmenCard}%
\linebreak[0]%
\begin{OmenCard}{bww}{Letter}{}
  \begin{CardStory}
    充滿潦草墨跡的鬼畫符。
  \end{CardStory}
  你可以將一個你的探險者標記交給另一名探險者。\smallbreak
  此後,在你的回合中,你可以移動到該名探險者所在的房間裡,並將\ThisName{}棄掉,該探險者亦將你的探險者標記棄掉。\smallbreak
\end{OmenCard}%
\linebreak[0]%
\begin{OmenCard}{bww}{Photograph}{}
  \begin{CardStory}
    當你看著這張照片時,它的景象漸漸變成這棟屋子裡的另一間房間。\\
    大概還是這棟屋子吧⋯
  \end{CardStory}
  當你探索新房間時,你可以將第一張抽到的房間板塊棄掉,重新抽取新的(合乎探索規則的)房間板塊,並進行探索。\smallbreak
\end{OmenCard}%
\linebreak[0]%
\begin{OmenCard}{bww}{Rope}{}
  \begin{CardStory}
    今天應該沒有用來吊死誰⋯吧?
  \end{CardStory}
  你可以從直接爬上\Room{Coal Chute},亦可從\Room{Ballroom}爬到\Room{Gallery}。\smallbreak
  當你經過\Room{Chasm}時,無需進行檢定。\smallbreak
  你不會因\Room{Collapsed Room}的效果受傷,並可從它下方的房間爬上來。\smallbreak
  當你使用升降機時,可以移動到任何的\Room{Landing}。\smallbreak
\end{OmenCard}%
\linebreak[0]%
\begin{OmenCard}{bww}{Vial}{}
  \begin{CardStory}
    一位老瘋婆說,你該喝下那瓶閃爍著琥珀色光芒的藥水。也許這就是她會變成這副德性的原因⋯
  \end{CardStory}
  此後每回合一次,你可擲2顆骰嘗試喝下琥珀色藥水:
  \begin{itemize}
    \item[4] 這不曉得是什麼血。獲得1級力量、1級速度、1級知識,但失去2級神志。
    \item[3] 你感到興奮。獲得1級物理屬性。
    \item[2] 實驗失敗,沒有效果。
    \item[1] 你做了個預知夢。受到2點精神傷害,並抽取1張道具卡。
    \item[0] 這有毒!受到2點物理傷害。
  \end{itemize}
\end{OmenCard}%
\linebreak[0]%
