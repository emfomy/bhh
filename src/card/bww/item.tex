%!TEX root = ../../../card.tex
%
\begin{ItemCard}{bww}{Blueprint}{}
  \begin{CardStory}
    這間在屋子的老舊地圖。
  \end{CardStory}
  當你執行移動相關的檢定時,例如進入、離開、或穿越房間時,你可以選擇檢定的結果。\smallbreak
  你可以在有升降機圖示的房間之間傳送移動。\smallbreak
\end{ItemCard}%
\linebreak[0]%
\begin{ItemCard}{bww}{Boomstick}{Weapon}
  \begin{CardStory}
    有點生鏽的槍身,\\
    看得出來有被好好的使用過,\\
    不過仍留下了兩發不錯的子彈。
  \end{CardStory}
  當你抽到此牌時,放置2枚\TokenPenta{Item}在這張牌上。\smallbreak
  你可以犧牲1枚\TokenPenta{Item},並使用\ThisName{}進行速度7攻擊,防禦方亦用速度進行防禦,並受到物理傷害。\smallbreak
  使用\ThisName{},你可對任何視線內(透過任何未受阻礙的門可看到)的對手進行攻擊,若攻擊失敗,你不會被反擊。\smallbreak
  當你用盡所有\TokenPenta{Item}時,棄掉\ThisName{}。\smallbreak
  \ThisName{}不得用於防禦。\smallbreak
  你不可同時使用多樣武器攻擊。\smallbreak
\end{ItemCard}%
\linebreak[0]%
\begin{ItemCard}{bww}{Camcorder}{}
  \begin{CardStory}
    看起來像是以前探險者所留下的,\\
    帶子裡還有一些影像,\\
    你應該好好把它看完。
  \end{CardStory}
  當你抽到此牌時,將與探險者人數等量的\TokenTri{Knowledge Roll}放置在這張牌上。\smallbreak
  當你持有\ThisName{}時,你可以拿取一枚\TokenTri{Knowledge Roll},並提升1級知識。\smallbreak
  你無法拿取超過一枚的\TokenTri{Knowledge Roll}。\smallbreak
  當所有\TokenTri{Knowledge Roll}都被拿完時,棄掉\ThisName{}。\smallbreak
\end{ItemCard}%
\linebreak[0]%
\begin{ItemCard}{bww}{Ceremonial Robe}{}
  \begin{CardStory}
    看不出來是祭司還是祭品的服裝。
  \end{CardStory}
  每回合一次,當你進行知識或神志檢定時,你可重擲任意數量的骰子。你必須使用重擲後的結果。\smallbreak
\end{ItemCard}%
\linebreak[0]%
\begin{ItemCard}{bww}{Chainsaw}{Weapon}
  \begin{CardStory}
    咈咈、寶貝、咈咈。
  \end{CardStory}
  當你使用\ThisName{}進行力量攻擊或防禦時,多擲1顆骰。\smallbreak
  你不可同時使用多樣武器攻擊。\smallbreak
\end{ItemCard}%
\linebreak[0]%
\begin{ItemCard}{bww}{Chalk}{}
  \begin{CardStory}
    由骨灰和蠟製作的簡單畫筆。
  \end{CardStory}
  你可以使用知識取代力量進行防禦。\smallbreak
\end{ItemCard}%
\linebreak[0]%
\begin{ItemCard}{bww}{Device}{}
  \begin{CardStory}
    由雜亂線圈與玻璃儀器所組成,\\
    用來測量這世上未知的東西。
  \end{CardStory}
  當你回合結束時,有任何對手與你處在同一房間之中,你提升1級知識。\smallbreak
  當你攻擊時,可棄掉此牌,多擲3顆骰,但此次攻擊只能用於搶奪物品。\smallbreak
\end{ItemCard}%
\linebreak[0]%
\begin{ItemCard}{bww}{Effigy}{}
  \begin{CardStory}
    精緻手工的公仔娃娃,\\
    穿著與你一樣的服裝。
  \end{CardStory}
  當你持有\ThisName{}時,所有非戰鬥時的\Roll{Trait}皆多擲1顆骰。\smallbreak
  若你失去\ThisName{},全屬性各失去1顆骰的等級。\smallbreak
\end{ItemCard}%
\linebreak[0]%
\begin{ItemCard}{bww}{Locket}{}
  \begin{CardStory}
    細細金鍊紀念著愛的回憶。
  \end{CardStory}
  立即獲得1級神志。\smallbreak
  若你失去\ThisName{},立即失去1級神志。\smallbreak
  當你受到精神傷害時,減少1點傷害。\smallbreak
\end{ItemCard}%
\linebreak[0]%
\begin{ItemCard}{bww}{Snake Oil}{}
  \begin{CardStory}
    破損的標籤上,\\
    模糊地註記著提升活力的效用。\\
    聞起來不像有毒的樣子。
  \end{CardStory}
  可自行使用,或用於治療同房的探險者。\smallbreak
  被\ThisName{}治療的探險者,將一項物理屬性和/或一項精神屬性恢復至初始值。\smallbreak
  使用後將\ThisName{}丟棄。\smallbreak
\end{ItemCard}%
\linebreak[0]%
\begin{ItemCard}{bww}{Teapot}{}
  \begin{CardStory}
    陶瓷的精巧茶壺,\\
    鑲嵌著粉紅色花紋的裝飾,\\
    顯露出高貴的質感。
  \end{CardStory}
  每回合一次,任何一位同房的探險者(包括你)受到物理傷害時,你可以抽取1張道具牌。\smallbreak
  若你身為/成為叛徒,將此牌放回道具牌堆洗勻。\smallbreak
\end{ItemCard}%
\linebreak[0]%
