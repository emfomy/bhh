%!TEX root = ../../../card.tex
%
\begin{EventCard}{bww}{Acupuncture}
	\begin{CardStory}
		房內點滿了令人愉悅的蠟燭,\\
		黑色長髮的女人穿著樸素外衣,\\
		用著溫暖的微笑歡迎你的到來。\\
		女人舉起一根在燭光下閃閃發光、\\
		比她頭髮還要細的長針,\\
		幾乎就要刺入你的眼裡…
	\end{CardStory}
	立即移至任一相鄰的房間並且失去1級神志,或是進行力量檢定嘗試反抗:
	\begin{itemize}
		\item[5+] 獲得1級力量、1級速度、1級神志。
		\item[3-4] 獲得1級力量和1級速度。
		\item[0-2] 失去1級力量和1級速度。
	\end{itemize}
\end{EventCard}%
\linebreak[0]%
\begin{EventCard}{bww}{Burial Mound}
	\begin{CardStory}
		散落的碎石掩埋著一具\\
		新鮮的屍體…太新鮮了…
	\end{CardStory}
	放置一枚\TokenSq{Burial Mound}在此房。\smallbreak
	作祟發生後,任何探險者離開此房間時皆受到死屍的攻擊。其右方的玩家代替死屍,進行力量4的攻擊。\smallbreak
	若探險者獲得 3+ 以上的勝利,移除\TokenSq{Burial Mound}。\smallbreak
\end{EventCard}%
\linebreak[0]%
\begin{EventCard}{bww}{Contract}
	\begin{CardStory}
		這裡有份以靈魂為代價,\\
		給予你力量的契約。\\
		好契約不簽嗎?
	\end{CardStory}
	你可自由選擇是否簽署此契約。\smallbreak
	若你決定簽署此契約,進行一次知識檢定以閱讀此文件:
	\begin{itemize}
		\item[5+] 你發現這份契約並未綁約,你成功留住了靈魂。獲得1級神志、並抽取1張道具牌。
		\item[4] 你發現了可以保住靈魂的小漏洞。抽取1張道具牌。
		\item[2-3] 你失去了靈魂,但獲得了力量。失去1級神志、並抽取1張道具牌。
		\item[0-1] 你的靈魂沒有價值,白簽了。失去1級神志。
	\end{itemize}
\end{EventCard}%
\linebreak[0]%
\begin{EventCard}{bww}{Flytrap}
	\begin{CardStory}
		一株巨大詭異的植物\\
		聳立在燈光及各種管線間。\\
		它的葉子枯黃、巨大的花莢半開,\\
		期待著營養的補給。\\
		它需要水分…你的血…
	\end{CardStory}
	立即受到1顆骰的物理傷害。\smallbreak
	放置一枚\TokenSq{Plant}在此房。\smallbreak
	作祟發生後,任何探險者或怪物進入此房間時,你可以代替捕蠅草,對該對手進行力量7的攻擊。\smallbreak
\end{EventCard}%
\linebreak[0]%
\begin{EventCard}{bww}{Ghost in The Machine}
	\begin{CardStory}
		一陣劈哩啪啦的聲響,\\
		伴隨著各種駭人的影像,\\
		在眼前的電視機內閃爍著。\\
		透過白花花的雜訊,\\
		你辨識出一個女孩,\\
		在這屋內某處玩耍。\\
		她停下動作並轉頭,\\
		向你招手示意加入他的遊戲。\\
		正當你還沒回過神來,\\
		電視螢幕已陷入一片黑暗…
	\end{CardStory}
	請立即進行神志檢定:
	\begin{itemize}
		\item[3+] 抽取一張頂樓的房間板塊,放入屋內,並移至該處。
		\item[0-2] 失去1點力量,並保留此牌一回合。到你下一個回合開始前,所有探險者的行動步數降至1格。
	\end{itemize}
\end{EventCard}%
\linebreak[0]%
\begin{EventCard}{bww}{Lightning Strikes}
	\begin{CardStory}
		從看到閃電時開始讀秒。兩秒。
	\end{CardStory}
	所有位於戶外或樓頂的探險者,立即進行力量檢定:
	\begin{itemize}
		\item[4+] 什麼事也沒發生。
		\item[1-3] 受到1點物理傷害。
		\item[0] 受到2點物理傷害。
	\end{itemize}
\end{EventCard}%
\linebreak[0]%
\begin{EventCard}{bww}{Misty Arch}
	\begin{CardStory}
		一座被彩色迷霧籠罩的拱門…
	\end{CardStory}
	你可自由選擇是否走進此拱門。\smallbreak
	若你決定走進此拱門,請立即擲3顆骰子:
	\begin{itemize}
		\item[5-6] 移至任何一間有尚未探索之門的房間,並抽取1張道具牌。
		\item[3-4] 移至任何一間有尚未探索之門的房間。
		\item[1-2] 你不知為何發生了變化。移至\Room{Entrance Hall},將你的角色卡翻面,並重置所有數值至初始值。
		\item[0] 移至\Room{Entrance Hall},並棄掉一張道具牌。
	\end{itemize}
\end{EventCard}%
\linebreak[0]%
\begin{EventCard}{bww}{Mutant Housepet}
	\begin{CardStory}
		從牆壁傳來一聲貓叫聲。\\
		真的是貓嗎?你這麼想著。\\
		接著從同一個位置又爆出了\\
		一聲嚎叫。難道兩隻貓?\\
		但搔抓的聲音又只有一個。\\
		難道他有兩個頭?\\
		你試著將牆面鑿開一探究竟…
	\end{CardStory}
	請立即進行一次速度檢定:
	\begin{itemize}
		\item[4+] 你成功躲開\ThisName{}的攻擊。獲得1點速度和1點神志。
		\item[0-3] \ThisName{}對你咬了一口。受到1顆骰的物理傷害。
	\end{itemize}
\end{EventCard}%
\linebreak[0]%
\begin{EventCard}{bww}{The Left Hand}
	\begin{CardStory}
		你的手開始發癢,然後燒了起來!\\
		疼動不斷蔓延,而你卻只能尖叫,\\
		並試圖把它與你的身體分離!\\
		但尖叫聲在喉嚨嘎然而止。\\
		突然,你發現自己沒有任何知覺,\\
		你試圖動動手指,卻發現那手\\
		彷彿有意識地向你的喉嚨靠近…
	\end{CardStory}
	請選擇一項:
	\begin{itemize}
		\item[•] 切除你的手。失去1點力量,並提升1點神志。
		\item[•] 替換你的手。失去1點力量,並抽取1張道具牌。
		\item[•] 保留你的手。失去2點神志,並提升1點力量。移往最鄰近有預兆圖示的房間,並抽取1張預兆牌。
	\end{itemize}
\end{EventCard}%
\linebreak[0]%
\begin{EventCard}{bww}{The Walls Have Eyes}
	\begin{CardStory}
		牆上掛著一排褪色的肖像畫,\\
		他們的眼睛看似跟隨著你的行動。\\
		難道這只是畫像的把戲?\\
		不,這些沈默的死人臉\\
		正試著告訴你什麼事…
	\end{CardStory}
	請立即進行一次神志檢定:
	\begin{itemize}
		\item[4+] 移至\Room{Arsenal},並抽取1張道具牌。(若\Room{Arsenal}尚未探索,將其找出並放進屋內。)
		\item[3] 獲得1級知識。
		\item[0-2] 失去1級神志。
	\end{itemize}
\end{EventCard}%
\linebreak[0]%
\begin{EventCard}{bww}{What Year is It!?}
	\begin{CardStory}
		你穿過那道門並停下腳步,\\
		發現自己似乎曾在幾年前來過。\\
		窗戶映射出你的樣貌,\\
		而你的臉龐看起來仍是那麼年輕。\\
		不過這次你早已不再天真…
	\end{CardStory}
	失去1點神志,並獲得1點力量和1點知識。\smallbreak
	若作祟已經發生,則另一方必須大聲朗讀他們劇本上的斜體字開場白及獲勝的故事文字。若作祟尚未發生,保留此牌直到作祟發生為止。\smallbreak
\end{EventCard}%
\linebreak[0]%
