%!TEX root = ../../../card.tex
%
\begin{OmenCard}{bhh}{Bite}{}
  \begin{CardStory}
    一聲咆哮,瀰漫著死亡的氣息,\\
    接續著痛苦、黑暗,然後消失⋯
  \end{CardStory}
  當你抽到此牌時,某種東西突然咬上了你!\smallbreak
  你的右方的玩家代替神秘生物,擲4顆骰子對你的攻擊,你以力量進行防禦。\smallbreak
  攻擊後,神秘生物又再一次隱沒在黑暗當中。\smallbreak
  \ThisName{}不可被棄置、交易、或搶奪。\smallbreak
\end{OmenCard}%
\linebreak[0]%
\begin{OmenCard}{bhh}{Book}{}
  \begin{CardStory}
    這是一本日記?實驗筆記?\\
    遠古手稿?還是一本妄語呢?
  \end{CardStory}
  立即獲得2級知識。\smallbreak
  若你失去\ThisName{},立即失去2級知識。\smallbreak
\end{OmenCard}%
\linebreak[0]%
\begin{OmenCard}{bhh}{Crystal Ball}{}
  \begin{CardStory}
    在球體中浮現了模糊的影像。
  \end{CardStory}
  在作祟發生後,每回合一次,你可進行知識檢定嘗試窺視\ThisName{}:
  \begin{itemize}
    \item[4+] 你看見真相。從道具牌堆中選擇一張牌,將牌堆洗勻後將選擇的牌放在牌堆頂端。
    \item[1-3] 你不自覺的地避開了視線。失去1級神志。
    \item[0] 你直視了地獄景象。失去2級神志。
  \end{itemize}
\end{OmenCard}%
\linebreak[0]%
\begin{OmenCard}{bhh}{Dog}{Companion}
  \begin{CardStory}
    看似友善的癩皮狗⋯\\
    至少你希望牠是⋯
  \end{CardStory}
  立即獲得1級力量和1級神志。\smallbreak
  若你失去\ThisName{},立即失去1級力量和1級神志。\smallbreak
  % 使用一個小怪物標示代表這隻\ThisName{},將牠放在你的房間裡。(不要和其他怪物搞混。)\smallbreak
  每回合一次,你可差遣\ThisName{}來回於六格內的房間(可以使用門和樓梯),途中牠可以攜帶一件物品去丟下,同時也可以撿拾並帶回一件掉落的物品。\smallbreak
  \ThisName{}不會被對手阻擋。牠不能通過單向的通道、或任何需要\RollAny{}的房間。牠也不能攜帶會減緩移動速度的物品。\smallbreak
  \ThisName{}不可被棄置、交易、或搶奪。\smallbreak
\end{OmenCard}%
\linebreak[0]%
\begin{OmenCard}{bhh}{Girl}{Companion}
  \begin{CardStory}
    被困住的孤獨女孩,你拯救了她!
  \end{CardStory}
  立即獲得1級神志和1級知識。\smallbreak
  若你失去\ThisName{},立即失去1級神志和1級知識。\smallbreak
  \ThisName{}不可被棄置、交易、或搶奪。\smallbreak
\end{OmenCard}%
\linebreak[0]%
\begin{OmenCard}{bhh}{Holy Symbol}{}
  \begin{CardStory}
    混沌世界中的和平象徵。
  \end{CardStory}
  立即獲得2級神志。\smallbreak
  若你失去\ThisName{},立即失去2級神志。\smallbreak
\end{OmenCard}%
\linebreak[0]%
\begin{OmenCard}{bhh}{Madman}{Companion}
  \begin{CardStory}
    口吐白沫、胡言亂語的瘋漢。
  \end{CardStory}
  立即獲得2級力量,並失去1級神志。\smallbreak
  若你失去\ThisName{},立即失去2級力量,並獲得1級神志。\smallbreak
  \ThisName{}不可被棄置、交易、或搶奪。\smallbreak
\end{OmenCard}%
\linebreak[0]%
\begin{OmenCard}{bhh}{Mask}{}
  \begin{CardStory}
    可以用來隱藏你意圖的陰沈面具。
  \end{CardStory}
  每回合一次,你可進行神志檢定嘗試使用\ThisName{}:
  \begin{itemize}
    \item[4+] 你可自由戴上或脫下\ThisName{}。
    \item[0-3] 你本回合無法使用\ThisName{}。
  \end{itemize}
  戴上面具後,立即獲得2級神志,並失去2級知識。\smallbreak
  脫下面具後,立即失去2級神志,並獲得2級知識。\smallbreak
\end{OmenCard}%
\linebreak[0]%
\begin{OmenCard}{bhh}{Medallion}{}
  \begin{CardStory}
    刻有五芒星標誌的徽章。
  \end{CardStory}
  你將不再受到\Room{Pentagram Chamber}、\Room{Crypt}、\Room{Graveyard}的影響。\smallbreak
\end{OmenCard}%
\linebreak[0]%
\begin{OmenCard}{bhh}{Ring}{}
  \begin{CardStory}
    一個磨損的戒指,\\
    上頭有著難以理解的刻記。
  \end{CardStory}
  當你遇攻擊的對手有神志屬性時,你可以使用\ThisName{}進行神志攻擊,嘗試造成精神傷害,對手亦須以神志進行防禦。\smallbreak
\end{OmenCard}%
\linebreak[0]%
\begin{OmenCard}{bhh}{Skull}{}
  \begin{CardStory}
    破碎且缺牙的骷髏頭。
  \end{CardStory}
  當你受到精神傷害時,可將所有的傷害轉為物理傷害,但不可只轉換部分傷害。\smallbreak
\end{OmenCard}%
\linebreak[0]%
\begin{OmenCard}{bhh}{Spear}{Weapon}
  \begin{CardStory}
    充滿能量而震動的武器。
  \end{CardStory}
  當你使用\ThisName{}進行力量攻擊時,多擲2顆骰。\smallbreak
  \ThisName{}不得用於防禦。\smallbreak
  你不可同時使用多樣武器攻擊。\smallbreak
\end{OmenCard}%
\linebreak[0]%
\begin{OmenCard}{bhh}{Spirit Board}{}
  \begin{CardStory}
    刻印著招喚死亡的文字與數字。
  \end{CardStory}
  此後每回合移動之前,你可以使用\ThisName{}窺視房間牌堆頂端的板塊。\smallbreak
  在作祟發生之後,每當你使用\ThisName{}時,叛徒可遣任意數量的怪物向你移動1格。若你身為叛徒,則忽略此效果。若沒有叛徒,則所有怪物皆須向你前進1格。\smallbreak
\end{OmenCard}%
\linebreak[0]%
