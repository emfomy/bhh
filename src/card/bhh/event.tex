%!TEX root = ../../../card.tex

\begin{EventCard}{bhh}{A Moment of Hope}{希望時刻}
  \begin{CardStory}
    房間湧現出不可思議的正面能量,\\
    似乎有什麼正努力對抗著\\
    這屋子裡的邪惡勢力。
  \end{CardStory}
  放置一枚\TokenSq{Blessing}在此房。\smallskip
  此後,任何英雄在此房內進行的\Roll{Trait},皆多擲1顆骰(至多8顆)。\smallskip
\end{EventCard}%
\linebreak[0]%
\begin{EventCard}{bhh}{Angry Being}{惱人的生物}
  \begin{CardStory}
    牠從牆上的黏液中冒出。\\
    而且⋯就在你身邊⋯
  \end{CardStory}
  請立即進行速度檢定:
  \begin{itemize}
    \item[5+] 你將他們遠遠甩開!獲得1級速度。
    \item[2-4] 受到1顆骰的精神傷害。
    \item[0-1] 受到1顆骰的精神傷害,和1顆骰的物理傷害。
  \end{itemize}
\end{EventCard}%
\linebreak[0]%
\begin{EventCard}{bhh}{Bloody Vision}{血腥幻象}
  \begin{CardStory}
    牆壁因血液而潮濕,\\
    鮮血從天花板不停滴下,\\
    流過牆壁、傢俱,直到你的鞋子⋯\\
    轉眼之間,又全都消失⋯
  \end{CardStory}
  請立即進行神志檢定:
  \begin{itemize}
    \item[4+] 你撐過去了!獲得1級神志。
    \item[2-3] 失去1級神志。
    \item[0-1] 你神志不清,突然攻擊1名跟你同房或相鄰房間的探險者或怪物(力量最低者)。
  \end{itemize}
\end{EventCard}%
\linebreak[0]%
\begin{EventCard}{bhh}{Burning Man}{燃燒之人}
  \begin{CardStory}
    全身著火的男子從房間跑過,\\
    濃膩的水泡在皮膚上冒出並破裂,\\
    最後只剩下炙熱的頭骨,\\
    頭顱撞擊至地面後,持續滾動⋯
  \end{CardStory}
  請立即進行神志檢定:
  \begin{itemize}
    \item[4+] 除了衣領下有些悶熱外,一切安好。獲得1級神志。
    \item[2-3] 快走,快走!你必須趕快離開!移至\Room{Entrance Hall}。
    \item[0-1] 你被烈焰所牽制,火花在你身上爆開。當你逃開時,受到1顆骰的物理傷害。
  \end{itemize}
\end{EventCard}%
\linebreak[0]%
\begin{EventCard}{bhh}{Closet Door}{壁櫥}
  \begin{CardStory}
    壁櫥的門半掩著,\\
    裡面肯定有什麼東西躲藏著⋯
  \end{CardStory}
  放置一枚\TokenSq{Closet}在此房。\\[0.5em]
  每回合一次,任何探險者皆可擲2顆骰子嘗試打開壁櫥:
  \begin{itemize}
    \item[4] 抽取1張道具卡。
    \item[2-3] 抽取1張事件卡。
    \item[0-1] 抽取1張事件卡,並移除標記。
  \end{itemize}
\end{EventCard}%
\linebreak[0]%
\begin{EventCard}{bhh}{Creepy Crawlies}{恐怖爬蟲}
  \begin{CardStory}
    成千上萬的爬蟲,\\
    落在你的皮膚上,\\
    鑽進衣服、頭髮之中⋯
  \end{CardStory}
  請立即進行神志檢定:
  \begin{itemize}
    \item[5+] 一眨眼,牠們全都消失了。獲得1級神志。
    \item[1-4] 失去1級神志。
    \item[0] 失去2級神志。
  \end{itemize}
\end{EventCard}%
\linebreak[0]%
\begin{EventCard}{bhh}{Creepy Puppet}{陰森傀儡}
  \begin{CardStory}
    散亂的雜物中,\\
    你發現一個令你毛骨悚然的傀儡。\\
    在你們四目相接時,\\
    他突然拿起小小的矛向你跳去⋯
  \end{CardStory}
  你的右方的玩家代替傀儡,對你的進行力量4的攻擊。\\[0.5em]
  若你因此受傷,除你以外擁有\Omen{Spear}的玩家,獲得2級力量。\\[0.5em]
\end{EventCard}%
\linebreak[0]%
\begin{EventCard}{bhh}{Debris}{瓦礫}
  \begin{CardStory}
    泥灰從天花板及牆上不斷掉落⋯
  \end{CardStory}
  請立即進行速度檢定:
  \begin{itemize}
    \item[3+] 你閃過了泥灰!獲得1級速度。
    \item[1-2] 你被瓦礫活埋!受到1顆骰的物理傷害。
    \item[0] 你被瓦礫活埋!受到2顆骰的物理傷害。
  \end{itemize}
  若被活埋,保留此卡,並無法再進行任何行動或使用物品,直到掙脫為止。\\[0.5em]
  此後每回合一次,你皆可進行力量檢定達 4+ 嘗試掙脫。其他探險者也能用此方法幫你掙脫。\\[0.5em]
  若檢定三次失敗,你自動掙脫並丟棄此卡,從下回合開始正常行動。\\[0.5em]
\end{EventCard}%
\linebreak[0]%
\begin{EventCard}{bhh}{Disquieting Sounds}{令人不安的聲響}
  \begin{CardStory}
    嬰兒的哭泣聲殷殷響起,\\
    控訴著被遺棄的命運。\\
    緊接著是一聲尖叫,\\
    以及刺耳的玻璃碎裂聲⋯\\
    接著,一切恢復寂靜⋯
  \end{CardStory}
  擲6顆骰子:
  \begin{itemize}
    \item 若結果大於或等於已被翻出的預兆卡數量,獲得1級神志。
    \item 反之,失去1級神志。
  \end{itemize}
\end{EventCard}%
\linebreak[0]%
\begin{EventCard}{bhh}{Drip... Drip... Drip...}{滴…滴…滴…}
  \begin{CardStory}
    一段有節奏的聲音刺激你的腦袋⋯
  \end{CardStory}
  放置一枚\TokenSq{Drip}在此房。\\[0.5em]
  此後,在此房間進行的任何\Roll{Trait},皆少擲1顆骰子(至少1顆)。\\[0.5em]
\end{EventCard}%
\linebreak[0]%
\begin{EventCard}{bhh}{Footsteps}{腳印}
  \begin{CardStory}
    地板沉緩地發出嘎軋的聲響,\\
    塵土向上飛揚,骯髒的地板上\\
    浮現了一步步的腳印。\\
    而就在它們快碰到你時,卻消失了⋯
  \end{CardStory}
  請立即擲1顆骰(如果你位於\Room{Crypt},多擲1顆):
  \begin{itemize}
    \item[4] 你和離你最近的探險者皆獲得1級力量。
    \item[3] 你獲得1級力量,但離你最近的探險者失去1級神志。
    \item[2] 你失去1級神志。
    \item[1] 你失去1級速度。
    \item[0] 所有探險者皆失去1級任意屬性。
  \end{itemize}
\end{EventCard}%
\linebreak[0]%
\begin{EventCard}{bhh}{Funeral}{葬儀}
  \begin{CardStory}
    你看見一口打開的棺材,\\
    而你正躺在裡面⋯
  \end{CardStory}
  請立即進行神志檢定:
  \begin{itemize}
    \item[4+] 一眨眼,全都消失了。獲得1級神志。
    \item[2-3] 幻覺使你心神不寧。失去1級神志。
    \item[0-1] 你真的躺在棺材裡!為了把自己從棺材裡挖出來,你失去1級力量和1級神志。\\
               若\Room{Crypt}或\Room{Graveyard}已被探索,移至其中之一。
  \end{itemize}
\end{EventCard}%
\linebreak[0]%
\begin{EventCard}{bhh}{Grave Dirt}{墳土飛揚}
  \begin{CardStory}
    這房間覆蓋著一層厚厚的墳土,\\
    塵土飄至皮膚,\\
    你吸入肺中而咳嗽⋯
  \end{CardStory}
  請立即進行力量檢定:
  \begin{itemize}
    \item[4+] 你將塵土拍落。獲得1級力量。
    \item[0-3] 事情不太對勁!保留此卡。每當你的回合開始時,受到1顆骰的物理傷害。
  \end{itemize}
  若你回合結束時,停留在\Room{Balcony}、\Room{Gardens}、\Room{Graveyard}、\Room{Gymnasium}、\Room{Larder}、\Room{Patio}、\Room{Tower},或是因道具卡而讓任一屬性升級時,丟棄此卡並回復正常。\\[0.5em]
\end{EventCard}%
\linebreak[0]%
\begin{EventCard}{bhh}{Groundskeeper}{園丁}
  \begin{CardStory}
    你轉身看見一名穿園丁服的男人,\\
    他舉起鏟子,朝你衝來。\\
    而就在鏟子快擊中你時,\\
    卻消失了,只留下滿地的泥濘⋯
  \end{CardStory}
  請立即進行知識檢定(如果你位於\Room{Gardens},少擲2顆):
  \begin{itemize}
    \item[4+] 你在泥濘中發現了某樣東西。抽取1張道具卡。
    \item[0-3] 園丁再次出現,並用鏟子用力揮向你的臉!你的右方的玩家代替園丁,對你的進行力量4的攻擊。
  \end{itemize}
\end{EventCard}%
\linebreak[0]%
\begin{EventCard}{bhh}{Hanged Men}{上吊之人}
  \begin{CardStory}
    一陣寒風吹過房間,三名男人\\
    被破爛麻繩吊掛在天花板上。\\
    他們靜靜地擺動者,\\
    用冰冷、死亡的眼神注視著你,\\
    接著逐漸化為塵土,\\
    令你感到窒息⋯
  \end{CardStory}
  請立即對所有屬性進行檢定:
  \begin{itemize}
    \item[2+] 該屬性不受影響。
    \item[0-3] 失去1級該屬性。
  \end{itemize}
  若所有屬性皆通過檢定,獲得1級任意屬性。\\[0.5em]
\end{EventCard}%
\linebreak[0]%
\begin{EventCard}{bhh}{Hideous Shriek}{駭人驚叫}
  \begin{CardStory}
    一開始聽起來像耳語,\\
    但最後卻變成了撕裂靈魂的驚叫⋯
  \end{CardStory}
  請所有探險者立即進行神志檢定:
  \begin{itemize}
    \item[4+] 你抵禦了這聲響。
    \item[1-3] 受到1顆骰的精神傷害。
    \item[0] 受到2顆骰的精神傷害。
  \end{itemize}
\end{EventCard}%
\linebreak[0]%
\begin{EventCard}{bhh}{Image in the Mirror}{鏡中影像}
  \begin{CardStory}
    這房間中,有一面古老的鏡子,\\
    你疑惑地看著鏡中自己的倒影,\\
    發現他不但自行動了起來,\\
    還帶著一臉恐懼。\\
    你瞭解到這是不同時空中的自己,\\
    而他需要幫助。於是你在鏡面寫下\\[0.5em]
    { \FontScript 這 \enskip 會 \enskip 有 \enskip 用 \enskip 的 }\\[0.5em]
    並透過鏡子遞給他一件物品⋯
\end{CardStory}
  若你手上沒有道具卡,則由下一位持有道具卡的玩家觸發此效果。\\[0.5em]
  選擇1張你所擁有的道具卡,洗入道具牌庫中。獲得1級知識。\\[0.5em]
\end{EventCard}%
\linebreak[0]%
\begin{EventCard}{bhh}{\scalebox{-1}[1]{Image in the Mirror}}{\scalebox{-1}[1]{鏡中影像}}
  \begin{CardStory}
    這房間中,有一面古老的鏡子,\\
    你驚恐地看著鏡中自己的倒影,\\
    發現他不但自行動了起來,\\
    還帶著一臉疑惑。\\
    你瞭解到這是不同時空中的自己,\\
    而他會幫助你。倒影在鏡面寫下\\[0.5em]
    \scalebox{-1}[1]{ \FontScript 這 \enskip 會 \enskip 有 \enskip 用 \enskip 的 }\\[0.5em]
    並透過鏡子遞給你一件物品⋯
\end{CardStory}
  抽取1張道具卡。\\[0.5em]
\end{EventCard}%
\linebreak[0]%
\begin{EventCard}{bhh}{It is Meant to Be}{命中註定}
  \begin{CardStory}
    你跌坐在地上,\\
    未來的影像閃過你的腦海⋯
  \end{CardStory}
  請選擇一項事件發生:
  \begin{itemize}
    \item[•] 窺視道具、事件、或預兆牌庫的前三張牌,依任意順序放回去,但不得說出看到的內容。
    \item[•] 擲4顆骰子並記錄結果。在未來任何擲骰機會,可直接使用這個結果,但不得超過該擲骰機會的最大可能值(僅限一次)。
  \end{itemize}
\end{EventCard}%
\linebreak[0]%
\begin{EventCard}{bhh}{Jonah’s Turn}{約拿的回合}
  \begin{CardStory}
    兩個小男孩在玩陀螺。\\
%     男孩:「約拿,你也想玩呀?」\\
%     約拿:「不,這只有我可以玩!」\\
%     語畢,約拿抓起陀螺,猛力砸向男孩的臉。影像逐漸模糊,依然可看見小男孩倒在血泊中,而約拿卻不停地砸向男孩的臉⋯
\end{CardStory}
  持有\Item{Puzzle Box}的探險者,必須立即將它丟棄,並抽取1張道具卡。\\[0.5em]
  若此事發生,你獲得1級神志,否則你失去1級神志。\\[0.5em]
\end{EventCard}%
\linebreak[0]%
\begin{EventCard}{bhh}{Lights Out}{燈熄了}
  \begin{CardStory}
    你的手電筒忽然熄滅。\\
    別擔心,總會有人帶著電池的⋯
  \end{CardStory}
  保留此卡,往後的回合你的行動步數降至1格。\\[0.5em]
  直到你回合結束時,有其他探險者與你處在同一房間之中,棄掉卡片並恢復正常移動。\\[0.5em]
  若你持有\Item{Candle},或在\Room{Furnace Room}結束回合,亦可棄掉卡片。\\[0.5em]
\end{EventCard}%
\linebreak[0]%
\begin{EventCard}{bhh}{Locked Safe}{上鎖的保險箱}
  \begin{CardStory}
    在肖像畫後方的牆上,\\
    你發現了一個隱藏的保險箱。\\
    當然,它肯定有某種陷阱⋯
  \end{CardStory}
  放置一枚\TokenSq{Safe}在此房。\\[0.5em]
  每回合一次,任何探險者皆可進行知識檢定嘗試打開保險箱:
  \begin{itemize}
    \item[5+] 你成功打開了保險箱,抽2張道具卡,並移除標記。
    \item[2-4] 受到1顆骰的物理傷害。
    \item[0-1] 受到2顆骰的物理傷害。
  \end{itemize}
\end{EventCard}%
\linebreak[0]%
\begin{EventCard}{bhh}{Mists From the Walls}{壁上霧靄}
  \begin{CardStory}
    霧氣從牆上湧現,\\
    煙霧之中浮現了滿滿的人臉⋯\\
    有些看來甚至不像人類⋯
  \end{CardStory}
  所有位於地下的玩家,立即進行神志檢定:
  \begin{itemize}
    \item[4+] 臉孔不過是光影的把戲罷了。
    \item[1-3] 受到1顆骰的物理傷害。若你位於有事件符號的房間中,多擲1顆骰。
    \item[0] 受到1顆骰的物理傷害。若你位於有事件符號的房間中,多擲2顆骰。
  \end{itemize}
\end{EventCard}%
\linebreak[0]%
\begin{EventCard}{bhh}{Mystic Slide}{神秘滑道}
  \begin{CardStory}
    你腳下的地板,忽然滑落並鬆開⋯
  \end{CardStory}
  若你位於地下,則由下一位不在地下玩家觸發此效果。\\[0.5em]
  放置一枚\TokenSq{Slide}在此房。\\[0.5em]
  請立即進行力量檢定嘗試控制滑道:
  \begin{itemize}
    \item[5+] 你成功控制住滑道。移至低一樓層的任意一間房間中。
    \item[0-4] 你失控了!抽取一張地下的房間板塊(若已無未探索的地下房間,則任選一間),跌落至該處,並受到1顆骰的物理傷害。若這是你不是的回合,則無須為新房間抽取事件/道具/預兆卡。
  \end{itemize}
  此後,任何探險者皆可透過檢定嘗試使用滑道,而不須花費行動步數。\\[0.5em]
\end{EventCard}%
\linebreak[0]%
\begin{EventCard}{bhh}{Night View}{暗夜景象}
  \begin{CardStory}
    一對幽魂情侶掠過你的面前,\\
    身穿結婚禮服無聲地漫步著⋯
  \end{CardStory}
  請立即進行知識檢定:
  \begin{itemize}
    \item[5+] 你瞭解他們曾經居住於此,你呼喚他們的名字,他們轉身向你走來,輕聲告訴你在這屋子裡隱藏的黑暗秘密。獲得1級知識。
    \item[0-4] 你恐懼地退,不敢直視。
  \end{itemize}
\end{EventCard}%
\linebreak[0]%
\begin{EventCard}{bhh}{Phone Call}{鈴鈴鈴⋯}
  \begin{CardStory}
    房內的電話忽然響起,\\
    你不自覺地接起這通電話⋯
  \end{CardStory}
  請立即擲2顆骰子,一名和藹的奶奶對你說:
  \begin{itemize}
    \item[4] 『茶點和蛋糕!茶點和蛋糕!你永遠是我的小可愛!』\\
獲得1級神志。
    \item[3] 『我一直都在你的身邊呀!小可愛,看著⋯』\\
獲得1級知識。
    \item[1-2] 『我在這,小可愛!給我一個吻吧!』\\
受到1顆骰的精神傷害。
    \item[0] 『小壞蛋必須受到處罰!』\\
受到2顆骰的物理傷害。
  \end{itemize}
\end{EventCard}%
\linebreak[0]%
\begin{EventCard}{bhh}{Possession}{著魔}
  \begin{CardStory}
    牆上分裂出數道黑影,\\
    你一時受到震驚而無法動作。\\
    黑影卻仍包圍著,\\
    一股惡寒從你心裡冷出來⋯
  \end{CardStory}
  請立即選擇任一種屬性進行檢定:
  \begin{itemize}
    \item[4+] 你成功抵禦了黑影的侵蝕,獲得1級該屬性。
    \item[0-3] 黑影吸取你的能量,該屬性被迫降至最低數值。如果該屬性已經是最低數值,則任選另一屬性降至最低。
  \end{itemize}
\end{EventCard}%
\linebreak[0]%
\begin{EventCard}{bhh}{Revolving Wall}{旋轉牆}
  \begin{CardStory}
    牆面旋轉至另一個地方⋯
  \end{CardStory}
  選定該房一面沒有門的牆壁,放置一枚\TokenSq{Wall Switch}在該牆壁上,並立即前往牆後的房間(若牆後無房間則為它抽1間)。\\[0.5em]
  開關雙面可用,此後每回合一次,任何探險者皆可進行知識檢定嘗試尋找旋轉牆的機關:
  \begin{itemize}
    \item[3+] 你發現了旋轉牆的機關。移動至牆後房間,而不須花費行動步數。
    \item[0-2] 你沒有辦法理解機關的奧妙。
  \end{itemize}
\end{EventCard}%
\linebreak[0]%
\begin{EventCard}{bhh}{Rotten}{腐敗}
  \begin{CardStory}
    這坊間的味道簡直糟透了!\\
    聞起來像死亡,\\
    有如屠宰場的鮮血⋯
  \end{CardStory}
  請立即進行神志檢定:
  \begin{itemize}
    \item[5+] 不過是惱人的氣味罷了。獲得1級神志。
    \item[2-4] 失去1級力量。
    \item[1] 失去1級力量和1級速度。
    \item[0] 你不停地作嘔。失去1級全屬性。
  \end{itemize}
\end{EventCard}%
\linebreak[0]%
\begin{EventCard}{bhh}{Secret Passage}{秘密通道}
  \begin{CardStory}
    牆壁的一部分滑了開來,\\
    一道發霉的通道正在等著你⋯
  \end{CardStory}
  放置一枚\TokenSq{Secret Passage}在此房。\\[0.5em]
  請立即擲3顆骰子,並放置另一枚標記於:
  \begin{itemize}
    \item[6] 任意一間房間中。
    \item[4-5] 任意一間樓上房間中。
    \item[2-3] 任意一間地面房間中。
    \item[0-1] 任意一間地下房間中。
  \end{itemize}
  你現在就可以使用祕密通道,就算你已耗盡行動步數亦可。\\[0.5em]
  此後,任何探險者皆可以使用通道,並花費1點行動步數。\\[0.5em]
\end{EventCard}%
\linebreak[0]%
\begin{EventCard}{bhh}{Secret Stairs}{秘密階梯}
  \begin{CardStory}
    一道毛骨悚然的聲響環繞著你,\\
    你發現了一個秘密階梯⋯
  \end{CardStory}
  放置一枚\TokenSq{Secret Stairs}在此房,並放置另一枚標記於另一樓層的任意一間房間中。\\[0.5em]
  你現在就可以使用祕密階梯,就算你已耗盡行動步數亦可。若你選擇現在使用階梯,到達新房間時,必須再抽取1張事件卡。\\[0.5em]
  此後,任何探險者皆可以使用通道,並花費1點行動步數。\\[0.5em]
\end{EventCard}%
\linebreak[0]%
\begin{EventCard}{bhh}{Shrieking Wind}{尖聲狂風}
  \begin{CardStory}
    風勢持續地增強,\\
    緩慢地漸強成刺耳的咆嘯聲⋯
  \end{CardStory}
  請所有位於\Room{Balcony}、\Room{Gardens}、\Room{Graveyard}、\Room{Patio}、\Room{Tower}、以及任何窗口向外房間的探險者,立即進行力量檢定:
  \begin{itemize}
    \item[5+] 穩住腳步
    \item[3-4] 強風將你擊倒。受到1顆骰的物理傷害。
    \item[1-2] 寒風注入你的心底。受到1顆骰的精神傷害。
    \item[0] 狂風狠狠地擊垮了你。受到1顆骰的物理傷害,同時選擇1件自己的物品,放到\Room{Entrance Hall}。
  \end{itemize}
\end{EventCard}%
\linebreak[0]%
\begin{EventCard}{bhh}{Silence}{寂靜}
  \begin{CardStory}
    在地底下,任何事物都變得沈默,\\
    彷彿連呼吸聲都消失了⋯
  \end{CardStory}
  請所有位於地下的探險者,立即進行神志檢定:
  \begin{itemize}
    \item[4+] 你冷靜地等待聽覺恢復。
    \item[1-3] 你發出一陣無聲的尖叫。受到1顆骰的精神傷害。
    \item[0] 你嚇壞了!受到2顆骰的精神傷害。
  \end{itemize}
\end{EventCard}%
\linebreak[0]%
\begin{EventCard}{bhh}{Skeletons}{骨骸}
  \begin{CardStory}
    即使成了骨骸,\\
    母親與孩子仍然緊緊擁抱著⋯
  \end{CardStory}
  放置一枚\TokenSq{Skeletons}在此房。\\[0.5em]
  每回合一次,任何探險者皆可進行神志檢定嘗試挖掘骨骸:
  \begin{itemize}
    \item[5+] 抽取一張道具卡,並移除標記。
    \item[0-4] 你在附近挖掘,卻一無所獲。受到1顆骰的精神傷害。
  \end{itemize}
\end{EventCard}%
\linebreak[0]%
\begin{EventCard}{bhh}{Smoke}{煙霧}
  \begin{CardStory}
    煙霧在你四周翻騰,\\
    你一邊咳嗽、一邊擦著眼淚⋯
  \end{CardStory}
  放置一枚\TokenSq{Smoke}在此房。\\[0.5em]
  煙霧會阻礙視線。\\[0.5em]
  此後在此房間進行的任何\Roll{Trait},皆少擲2顆骰子(至少1顆)。\\[0.5em]
\end{EventCard}%
\linebreak[0]%
\begin{EventCard}{bhh}{Something Hidden}{隱藏著什麼⋯}
  \begin{CardStory}
    這房間似乎有些古怪,\\
    但到底會是什麼呢?\\
    這問題持續困擾著你⋯
  \end{CardStory}
  請立即進行知識檢定以找出原因:
  \begin{itemize}
    \item[4+] 牆上的一部分滑了開來,你發現了一個壁櫥。抽取1張道具卡。
    \item[0-3] 仍然找不出原因,使你發狂!失去1級神志。
  \end{itemize}
\end{EventCard}%
\linebreak[0]%
\begin{EventCard}{bhh}{Something Slimy}{黏滑之物}
  \begin{CardStory}
    在你腳踝上的是什麼東西?\\
    是一隻蟲?是誰的觸角?\\
    還是一隻死人的手⋯
  \end{CardStory}
  請立即進行速度檢定:
  \begin{itemize}
    \item[4+] 你很快地掙脫。獲得1級速度。
    \item[1-3] 失去1級力量。
    \item[0] 失去1級力量和1級速度。
  \end{itemize}
\end{EventCard}%
\linebreak[0]%
\begin{EventCard}{bhh}{Spider}{蜘蛛}
  \begin{CardStory}
    拳頭般大小的蜘蛛,\\
    突然落在你的肩膀上,\\
    爬入你的頭髮⋯
  \end{CardStory}
  請立即進行速度檢定嘗試撥開他,或是進行神志檢定嘗試保持冷靜:
  \begin{itemize}
    \item[4+] 牠離開了。獲得1級該屬性。
    \item[1-3] 牠咬了你!受到1顆骰的物理傷害。
    \item[0] 牠大口地咬去你一塊肉!受到2顆骰的物理傷害。
  \end{itemize}
\end{EventCard}%
\linebreak[0]%
\begin{EventCard}{bhh}{The Beckoning}{呼喚}
  \begin{CardStory}
    離開!我必須離開!飛向自由吧!
  \end{CardStory}
  請所有位於\Room{Balcony}、\Room{Gardens}、\Room{Graveyard}、\Room{Tower}、以及任何窗口向外的房間的探險者,立即進行神志檢定:</Roll0>
  \begin{itemize}
    \item[3+] 你從邊緣退開。
    \item[0-2] 你縱身跳進了\Room{Patio},將人物移到此處,並受到1顆骰的物理傷害。(若\Room{Patio}尚未探索,將其找出並放進屋內。)
  \end{itemize}
\end{EventCard}%
\linebreak[0]%
\begin{EventCard}{bhh}{The Lost One}{迷失之魂}
  \begin{CardStory}
    一名身穿美國內戰晚禮服的女子,\\
    正揮手招喚著你,\\
    不久後,你便陷入了催眠中⋯
  \end{CardStory}
  請立即進行知識檢定:
  \begin{itemize}
    \item[5+] 再次清醒過來。獲得1級知識。
    \item[0-4] 另外擲3顆骰,幽魂引導你至:
      \begin{itemize}
        \item[6] \Room{Entrance Hall}。
        \item[4-5] \Room{Upper Landing}。
        \item[2-3] 抽取一張樓上的房間板塊。
        \item[0-1] 抽取一張地下的房間板塊。
      \end{itemize}
      若已無未探索房間板塊,則直接前往\Room{Entrance Hall}。
  \end{itemize}
\end{EventCard}%
\linebreak[0]%
\begin{EventCard}{bhh}{The Voice}{人聲}
  \begin{CardStory}
    「我在地底下,被埋在地底下⋯」\\
    微弱的聲音只說出這樣一句,\\
    便不再發聲⋯
  \end{CardStory}
  請立即進行知識檢定:
  \begin{itemize}
    \item[4+] 你在地板下發現了什麼。抽取1張道具卡。
    \item[0-3] 你依循著聲音挖開地板,卻一無所獲。
  \end{itemize}
\end{EventCard}%
\linebreak[0]%
\begin{EventCard}{bhh}{The Walls}{牆}
  \begin{CardStory}
    在這間溫暖的房間中,\\
    活生生的牆壁有如心跳得正動,\\
    你的心臟也跟著這屋子的節奏拍打。忽然間,你被吸入牆壁之中⋯
  \end{CardStory}
  請立即抽取一張新的房間板塊,並移到此處。\\[0.5em]
\end{EventCard}%
\linebreak[0]%
\begin{EventCard}{bhh}{Webs}{蜘蛛網}
  \begin{CardStory}
    一如往常,你伸手把蜘蛛網撥開,\\
    但這次沒能成功,你被黏住了⋯
  \end{CardStory}
  請立即進行力量檢定:
  \begin{itemize}
    \item[4+] 你掙脫了,獲得1級力量。
    \item[0-3] 你被黏住了。
  \end{itemize}
  若被黏住,保留此卡,並無法再進行任何行動或使用物品,直到掙脫為止。\\[0.5em]
  此後每回合一次,你皆可進行力量檢定達 4+ 嘗試掙脫。其他探險者也能用此方法幫你掙脫。檢定失敗的探險者,回合立即結束。\\[0.5em]
  若檢定三次失敗,你自動掙脫並丟棄此卡,從下回合開始正常行動。\\[0.5em]
\end{EventCard}%
\linebreak[0]%
\begin{EventCard}{bhh}{What the...?}{咦?}
  \begin{CardStory}
    當你回頭轉向經過的房間時,\\
    卻什麼也沒看到,\\
    只留下一層迷霧⋯
  \end{CardStory}
  請將你身處的房間板塊移至同一樓層中,任一尚未探索的門後,房間上的所有標記皆一起移動。\\[0.5em]
\end{EventCard}%
\linebreak[0]%
\begin{EventCard}{bhh}{WHOOPS!}{哎呀!}
  \begin{CardStory}
    你發現腳下似乎有什麼東西,\\
    正要躲開時,卻被牠狠狠擊倒,\\
    只聽見一陣咯咯笑聲離你遠去⋯
  \end{CardStory}
  請將你所有的道具卡(不包含預兆卡)覆蓋於桌面,由你右邊的玩家隨機選擇丟棄其中一張。\\[0.5em]
\end{EventCard}%
