%!TEX root = ../../../card.tex

\begin{ItemCard}{bhh}{Adrenaline Shot}
  \begin{CardStory}
    一管含奇怪螢光液體的針筒。
  \end{CardStory}
  每當你進行任何檢定前,可注入此劑,使你擲出的結果增加4點。 \smallskip
  使用後將\ThisName{}丟棄。 \smallskip
\end{ItemCard}%
%
\begin{ItemCard}{bhh}{Amulet Of The Ages}
  \begin{CardStory}
    鑲嵌著寶石的古代銀器,\\
    上頭銘刻著祝福。
  \end{CardStory}
  立即獲得1級全屬性。\smallskip
  若你失去\ThisName{},立即失去3級全屬性。\smallskip
\end{ItemCard}%
%
\begin{ItemCard}{bhh}{Angel Feather}
  \begin{CardStory}
    你手中飄動著一根完美的羽毛。
  \end{CardStory}
  當你進行任何擲骰前,可以使用\ThisName{}並指定一個 0-8 的數字,視為擲骰結果。\smallskip
  使用後將\ThisName{}丟棄。\smallskip
\end{ItemCard}%
%
\begin{ItemCard}{bhh}{Armor}
  \begin{CardStory}
    雖然只是文藝復興博覽會的道具,\\
    但仍然是金屬製。
  \end{CardStory}
  每當你受到任何物理傷害時,由於\ThisName{}的保護,減少1點物理傷害。\smallskip
  \ThisName{}不會被搶奪。\smallskip
\end{ItemCard}%
%
\begin{ItemCard}{bhh}{Axe}% usage=weapon
  \begin{CardStory}
    非常尖銳。
  \end{CardStory}
  當你使用\ThisName{}進行力量攻擊時,多擲1顆骰。\smallskip
  \ThisName{}不得用於防禦。\smallskip
  你不可同時使用多樣武器攻擊。\smallskip
\end{ItemCard}%
%
\begin{ItemCard}{bhh}{Bell}
  \begin{CardStory}
    能發出宏亮聲響的黃銅鈴鐺?
  \end{CardStory}
  立即獲得1級神志。\smallskip
  若你失去\ThisName{},立即失去1級神志。\smallskip
  在作祟發生後,每回合一次,你可進行神志檢定嘗試搖響\ThisName{}:
  \begin{itemize}
    \item[5+] 你可將任意數量且未受阻礙的英雄向你拉進1格。
    \item[0-4] 叛徒可遣任意數量的怪物向你移動1格。若你身為叛徒,則忽略此效果。若沒有叛徒,則所有怪物皆須向你前進1格。
  \end{itemize}
\end{ItemCard}%
%
\begin{ItemCard}{bhh}{Blood Dagger}% usage=weapon
  \begin{CardStory}
    一把齷齪的武器,\\
    針管自劍柄延伸出,\\
    深深地插入你的靜脈中⋯
  \end{CardStory}
  當你使用\ThisName{}進行力量攻擊時,多擲3顆骰(至多8顆),並失去1級速度。\smallskip
  \ThisName{}不得用於防禦。\smallskip
  你不可同時使用多樣武器攻擊。\smallskip
  \ThisName{}不可被棄置、交易。\smallskip
  若\ThisName{}被搶奪,受到2顆骰的物理傷害。\smallskip
\end{ItemCard}%
%
\begin{ItemCard}{bhh}{Bottle}
  \begin{CardStory}
    不透明的藥罐,\\
    流動著黑色的液體。
  \end{CardStory}
  在作祟發生後,每回合一次,你可擲3顆骰嘗試喝下黑色液體:
  \begin{itemize}
    \item[6] 移至任一間房間。
    \item[5] 獲得2級力量和2級速度。
    \item[4] 獲得2級神志和2級知識。
    \item[3] 獲得1級知識,並失去1級力量。
    \item[2] 失去2級神志和2級知識。
    \item[1] 失去2級力量和2級速度。
    \item[0] 失去2級全屬性。
  \end{itemize}
  使用後將\ThisName{}丟棄。\smallskip
\end{ItemCard}%
%
\begin{ItemCard}{bhh}{Candle}
  \begin{CardStory}
    能夠驅散陰影⋯\\
    至少你希望它真能如此⋯
  \end{CardStory}
  當你因事件而進行檢定時,多擲1顆骰(至多8顆)。\smallskip
  若你同時持有\Item{Bell}、\Omen{Book}、\Item{Candle}時,立即獲得2級全屬性。但若你失去任何一樣,立即失去2級全屬性。\smallskip
\end{ItemCard}%
%
\begin{ItemCard}{bhh}{Dark Dice}
  \begin{CardStory}
    你認為你夠幸運嗎?
  \end{CardStory}
  每回合一次,你可擲3顆骰嘗試使用\ThisName{}:
  \begin{itemize}
    \item[6] 移至任一英雄所在的房間。
    \item[5] 將你與所有同房的探險者,移至任一相鄰的房間。
    \item[4] 獲得1級物理屬性。
    \item[3] 移至任一相鄰的房間。
    \item[2] 獲得1級精神屬性。
    \item[1] 抽取1張事件卡。
    \item[0] 將所有屬性降至最低數值,並丟棄此卡。
  \end{itemize}
\end{ItemCard}%
%
\begin{ItemCard}{bhh}{Dynamite}
  \begin{CardStory}
    引信尚未點燃⋯
  \end{CardStory}
  在你的一次攻擊中,你可以使用\ThisName{}取代攻擊,將\ThisName{}點然並扔進一間相鄰的房間。該房間的所有探險者與怪物,皆需進行速度檢定:
  \begin{itemize}
    \item[5+] 你躲開了炸藥的攻擊。
    \item[0-4] 受到4點物理傷害。
  \end{itemize}
  使用後將\ThisName{}丟棄。\smallskip
\end{ItemCard}%
%
\begin{ItemCard}{bhh}{Healing Salve}
  \begin{CardStory}
    淺碗中裝著粘稠的糊狀物。
  \end{CardStory}
  可自行使用,或用於治療同房的探險者。\smallskip
  被\ThisName{}治療的探險者,將力量和速度恢復至初始值。\smallskip
  使用後將\ThisName{}丟棄。\smallskip
\end{ItemCard}%
%
\begin{ItemCard}{bhh}{Idol}
  \begin{CardStory}
    祂選擇你或許是為了某些目的⋯\\
    像是活體獻祭⋯
  \end{CardStory}
  每回合一次,當你進行任何檢定、戰鬥、或事件擲骰前,可摩擦\ThisName{}並犧牲1級神志,多擲2顆骰(至多8顆)。\smallskip
\end{ItemCard}%
%
\begin{ItemCard}{bhh}{Lucky Stone}
  \begin{CardStory}
    光滑、看似平凡無奇的石頭,\\
    你感覺到它似乎可為你帶來好運。
  \end{CardStory}
  在你的任何擲骰後,可摩擦\ThisName{},並重擲任意數量的骰子。\smallskip
  使用後將\ThisName{}丟棄。\smallskip
\end{ItemCard}%
%
\begin{ItemCard}{bhh}{Medical Kit}
  \begin{CardStory}
    一名醫生的急救包,\\
    某些重要資源已被用盡。
  \end{CardStory}
  每回合一次,你可進行知識檢定嘗試使用\ThisName{}治療自己或同房的探險者:
  \begin{itemize}
    \item[8+] 恢復3級物理屬性。
    \item[6-7] 恢復2級物理屬性。
    \item[4-5] 恢復1級物理屬性。
    \item[0-3] 你無法理解\ThisName{}該如何使用。
  \end{itemize}
  \ThisName{}無法將屬性提升超過初始值。\smallskip
\end{ItemCard}%
%
\begin{ItemCard}{bhh}{Music Box}
  \begin{CardStory}
    手製的舊式音樂盒,\\
    在你腦海中繚繞一陣幽森的旋律。
  \end{CardStory}
  每回合一次,你可開啟或關閉\ThisName{}。\smallskip
  當\ThisName{}被開啟後,房內所有的探險者皆須進行神志檢定達 4+,若失敗,則被旋律催眠,並立即結束回合。\smallskip
  當持有\ThisName{}被催眠時,\ThisName{}掉落至地面,並保持開啟狀態。\smallskip
\end{ItemCard}%
%
\begin{ItemCard}{bhh}{Pickpocket’s Glove}
  \begin{CardStory}
    幫助自己從未如此容易。
  \end{CardStory}
  當你與其他探險者處於同一房間時,可以使用\ThisName{},任意偷取該探險者所持有的其中一項物品。\smallskip
  使用後將\ThisName{}丟棄。\smallskip
\end{ItemCard}%
%
\begin{ItemCard}{bhh}{Puzzle Box}
  \begin{CardStory}
    應該有辦法打開吧?
  \end{CardStory}
  每回合一次,你可進行知識檢定嘗試打開\ThisName{}:
  \begin{itemize}
    \item[6+] 你順利地打開\ThisName{}。抽取2張道具卡,並丟棄此卡。
    \item[0-5] 你無法理解\ThisName{}的奧妙。
  \end{itemize}
\end{ItemCard}%
%
\begin{ItemCard}{bhh}{Rabbit’s Foot}
  \begin{CardStory}
    帶來幸運的左後腳,\\
    但那隻兔子就不怎麼幸運了⋯
  \end{CardStory}
  每回合一次,你可重擲一顆骰子。你必須使用重擲後的結果。\smallskip
\end{ItemCard}%
%
\begin{ItemCard}{bhh}{Revolver}% usage=weapon
  \begin{CardStory}
    外型老舊,\\
    但看似極具殺傷力的武器。
  \end{CardStory}
  你可以使用\ThisName{}進行速度攻擊,防禦方亦用速度進行防禦,並受到物理傷害。\smallskip
  當你使用\ThisName{}進行速度攻擊,多擲1顆骰(至多8顆)。\smallskip
  使用\ThisName{},你可對任何視線內(透過任何未受阻礙的門可看到)的對手進行攻擊,若攻擊失敗,你不會被反擊。\smallskip
  \ThisName{}不得用於防禦。\smallskip
  你不可同時使用多樣武器攻擊。\smallskip
\end{ItemCard}%
%
\begin{ItemCard}{bhh}{Sacrificial Dagger}% usage=weapon
  \begin{CardStory}
    彎曲的鐵製短劍,\\
    有著神秘的符號與斑斑的血跡。
  \end{CardStory}
  當你使用\ThisName{}進行力量攻擊時,多擲2顆骰(至多8顆),但請先進行知識檢定判定副作用:
  \begin{itemize}
    \item[6+] 沒有副作用。
    \item[3-5] 受到1顆骰的精神傷害。
    \item[0-2] 這把匕首倒插入你的手中!受到2顆骰的物理傷害,並且攻擊失敗,且本回合再也無法進行攻擊。
  \end{itemize}
  \ThisName{}不得用於防禦。\smallskip
  你不可同時使用多樣武器攻擊。\smallskip
\end{ItemCard}%
%
\begin{ItemCard}{bhh}{Smelling Salts}
  \begin{CardStory}
    深深吸了一口⋯
  \end{CardStory}
  可自行使用,或用於治療同房的探險者。\smallskip
  被\ThisName{}治療的探險者,將知識恢復至初始值。\smallskip
  使用後將\ThisName{}丟棄。\smallskip
\end{ItemCard}%
